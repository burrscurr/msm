\begin{definition}{Detailled Balance, Reversibilität}{mk-rev}
Sei $(X)_{n\in N_0}$ eine Markovkette mit Zustandsraum $S$, Übergangsmatrix
$\Pi$ und $\pi$ eine \link{def:mk-vert}{Verteilung}. Dann heißt $\pi$
\defw{reversibel}, wenn gilt:
\[
\forall x,y\in S:\ \pi(y)\cdot\Pi(y,x) = \pi(x)\cdot\Pi(x,y)
\]
Diese Bedingung wird auch als \defw{detailled Balance} bzw. als
\defw{detailliertes Gleichgewicht} bezeichnet.

Eine Markovkette mit reversibler Verteilung wird als \defw{reversibel}
bezeichnet.
\end{definition}

In einer reversiblen Markovkette kann also nicht unterschieden werden, ob der
Prozess vorwärts oder rückwärts abläuft\more{yt-detailled-balance}. Für jedes
Zustandspaar wird – intuitiv ausgedrückt – in beiden Richtungen die gleiche
Wahrscheinlichkeitsdichte übertragen.

Liegt für eine Verteilung ein solches Gleichgewicht (detailled balance) vor,
ändert sich die Verteilung auch nach zusätzlichen Zeitschritten nicht.
Deshalb gilt folgender Satz:

\begin{theorem}{}{mk-db}
Sei $\pi$ eine Verteilung einer Markovkette, die die Bedingung der
detailled balance erfüllt. Dann ist $\pi$ eine
\link{def:mk-stat}{stationäre Verteilung}.
\end{theorem}

Reversibilität ist eine stärkere Bedinung als Stationarität, es gibt also
stationäre Verteilungen, die nicht die Bedingung der detaillierten Balance
erfüllen.

\begin{theorem}{Schnittprinzip}{mk-schnittp}
Sei $(X)_{n\in N_0}$ eine \link{def:mk-ergod}{ergodische} Markovkette. Ist der
\link{def:mk-igraph}{Interaktionsgraph} von $(X)$ linear, das jeder Schnitt
zwischen zwei verbundenen Zuständen zerlegt den Graph in zwei disjunkte und
unverbundene Teile, so ist die stationäre Verteilung immer reversibel.
\end{theorem}

Da wir wissen, dass eine ergodische Markovkette immer genau eine stationäre
Verteilung besitzt (Satz \ref{satz:ergod}), folgt aus dem Schnittprinzip, dass
diese Verteilung auch reversibel ist. Dann können wir für die Verteilung die
detaillierten Balance annehmen. Das ist besonders hilfreich, wenn der
betrachtete Prozess sehr viele Zustände besitzt und die stationäre Verteilung
aufwändig zu berechnen ist.

\begin{example}{Warteschlange}{}
Ein bestimmter Server kann bis zu 100 Anfragen parallel bearbeiten. Pro Sekunde
erhält der Server mit Wahrscheinlichkeit $p^+ = 0.2$ eine neue Anfrage und schließt
mit Wahrscheinlichkeit $p^-=0.21$ eine Anfrage ab. Werden bereits 100 Anfragen
bearbeitet, werden neue Anfragen abgewiesen.

\emph{Wie hoch ist die Wahrscheinlichkeit, dass ein ankommender Auftrag abgewiesen
wird? Wie hoch ist die Wahrscheinlichkeit, dass die Maschine leerläuft?}

Die Anzahl der aktiven Anfragen kann durch eine Markovkette mit Zustandsraum
$S=\{0, 1,\ldots, 100\}$ beschrieben werden:

\medskip
\begin{tikzpicture}
  \node[v] (0) at (0,1) {0};
  \node[v] (1) at (2,1) {1};
  \node[v] (2) at (4,1) {2};
  \node[v] (n) at (6, 1) {$\ldots$};
  \node[v] (99) at (8,1) {99};
  \node[v] (100) at (10,1) {100};

  \draw[e] (0) to[loop left] node[anchor=south] {$p^-$} ();
  \draw[e] (100) to[loop right] node[anchor=south] {$p^+$} ();
  \draw[e] (0) to[bend left] node[anchor=south] {$p^+$} (1);
  \draw[e] (1) to[bend left] node[anchor=north] {$p^-$} (0);
  \draw[e] (1) to[bend left] node[anchor=south] {$p^+$} (2);
  \draw[e] (2) to[bend left] node[anchor=north] {$p^-$} (1);
  \draw[e] (2) to[bend left] node[anchor=south] {$p^+$} (n);
  \draw[e] (n) to[bend left] node[anchor=north] {$p^-$} (2);
  \draw[e] (n) to[bend left] node[anchor=south] {$p^+$} (99);
  \draw[e] (99) to[bend left] node[anchor=north] {$p^-$} (n);
  \draw[e] (99) to[bend left] node[anchor=south] {$p^+$} (100);
  \draw[e] (100) to[bend left] node[anchor=north] {$p^-$} (99);
\end{tikzpicture}

Die Markovkette besteht nur aus einer einzigen Klasse, ist daher abgeschlossen,
rekurrent und irreduzibel. Da die Zustände 0 und 100 Periode 1 besitzten, ist
die Markovkette aperiodisch. Folglich ist die Markovkette auch ergodisch, es
existiert also eine eindeutige stationäre Verteilung $\stat$.

Diese Verteilung ist wegen der 101 Zustände sehr aufwändig zu berechnen. Wir
können jedoch das Schnittprinzip (Satz \ref{satz:mk-schnittp}) anwenden, da der
Interaktionsgraph der Markovkette linear ist. Das bedeutet, dass für die
stationäre Wahrscheinlichkeit die Bedingung der detaillierten Balance gilt:
\begin{align*}
\forall x,y\in S:\ &\pi(y)\cdot\Pi(y,x) = \pi(x)\cdot\Pi(x,y) \\
\implies&\stat(i)\cdot p^+ = \stat(i+1)\cdot p^- \\
\implies&\stat(i+1) = \frac{p^+}{p^-}\cdot\stat(i) \\
\implies&\stat = (c, c\cdot\gamma,c\cdot\gamma^2,\ldots,
c\cdot\gamma^{100})^{\T},\ \gamma = \frac{p^+}{p^-}, c\in\R
\end{align*}
Da $\stat$ eine Verteilung ist, wissen wir dass die Einträge insgesamt 1
ergeben. Damit kann $c$ bestimmt werden:
\begin{align*}
1 &= \sum_{i=0}^{100}c\cdot\gamma^i = c \sum_{i=0}^{100}\gamma^i \\
\overset{\text{Geom. Summenformel}}{\iff} 1 &= c\cdot\frac{1-\gamma^{101}}{1-\gamma} \\
\implies \stat(i) &= c\cdot\gamma^i = \frac{1-\gamma}{1-\gamma^{101}}\cdot \gamma^i
\end{align*}

Die Wahrscheinlichkeit, dass ein ankommender Auftrag abgewiesen wird, ist
$\stat(100)=3.6\cdot10^{-4}$, der Server also bereits voll ausgelastet ist. Die
Wahrscheinlichkeit für Leerlauf ist $\stat(0) = 4.79\cdot10^{-2}$.
\end{example}
