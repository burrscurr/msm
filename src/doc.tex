\documentclass{report}

\usepackage[utf8]{inputenc}
\usepackage[T1]{fontenc}
\usepackage[ngerman]{babel}

\usepackage{hyphenat}
\hyphenation{Mathe-matik}

% Seitenlayout
\usepackage[a4paper,width=135mm,top=25mm,bottom=25mm,bindingoffset=0mm]{geometry}

% Bilder/Darstellungen
\usepackage{graphicx}
\graphicspath{{./img/}}
\usepackage{float}
\usepackage{wrapfig}

% Graphen
\usepackage{tikz}
\tikzset{
    v/.style={circle,minimum size=2ex, text width=5mm, align=center, draw=black},
    e/.style={-stealth, thick}}

% Mathematikfoo
\usepackage{amsmath}
\usepackage{amsthm}
\usepackage{amssymb}
\usepackage{dsfont}   % \mathds{1}

% Algorithmen
\usepackage[
german,
ruled
]{algorithm2e}

% Verlinkungen
\usepackage{hyperref}
\usepackage{cleveref}

% Literatur
\usepackage{biblatex}
\addbibresource{ref/zufallszahlen.bib}
\addbibresource{ref/markov.bib}

% Einrahmungen
\usepackage[most,many]{tcolorbox}

% Layout von https://tex.stackexchange.com/questions/369430/
% Definitionen
\tcbset{definitionstyle/.style={
    enhanced,
    sharp corners,
    attach boxed title to top left={
      xshift=-1mm,
      yshift=-4mm,
      yshifttext=-3mm
    },
    top=1.5ex,
    colback=white,
    colframe=green!40!black,
    fonttitle=\bfseries,
    boxed title style={
      sharp corners,
    size=small,
    colback=green!40!black,
    colframe=green!40!black
  }
}}
\newtcbtheorem{definition}{Definition}{definitionstyle}{def}

% Sätze
\tcbset{theoremstyle/.style={
    enhanced,
    sharp corners,
    attach boxed title to top left={
      xshift=-1mm,
      yshift=-4mm,
      yshifttext=-3mm
    },
    top=1.5ex,
    colback=white,
    colframe=blue!95!black,
    fonttitle=\bfseries,
    boxed title style={
      sharp corners,
    size=small,
    colback=blue!95!black,
    colframe=blue!95!black,
  }
}}

\newtcbtheorem{theorem}{Satz}{theoremstyle}{satz}

% Beispiele
\tcbset{examplestyle/.style={
    enhanced,
    sharp corners,
    attach boxed title to top left={
      xshift=-1mm,
      yshift=-4mm,
      yshifttext=-3mm
    },
    top=1.5ex,
    colback=white,
    colframe=white!55!black,
    fonttitle=\bfseries,
    boxed title style={
      sharp corners,
    size=small,
    colback=white!50!black,
    colframe=white!50!black,
  }
}}
\newtcbtheorem{example}{Beispiel}{examplestyle}{bsp}

% Lemmata
\newtheoremstyle{lemmastyle}
  {0.5}% space above
  {1}% space below
  {\itshape}% body font
  {}% indent
  {\textbf}%
  {}%
  { }% no newline after 'Lemma'
  {\thmname{#1}.}% don't print a lemma number
\theoremstyle{lemmastyle}
\newtheorem{lemma}{Lemma}


% Abkürzungen für Mathezeugs
\newcommand{\N}{\mathbb{N}}
\newcommand{\R}{\mathbb{R}}
\newcommand{\e}{\mathrm{e}}
\newcommand{\lr}{\leftrightarrow}
\newcommand{\pd}[2]{\frac{\partial #1}{\partial #2}} % partielle Ableitungen
\newcommand{\zvec}[1]{\overrightarrow{\mathrm{#1}}}
\newcommand{\T}{\top}                % Matrixtransposition
\DeclareMathOperator{\E}{E}          % Erwartungswert
\DeclareMathOperator{\Var}{Var}      % Varianz
\DeclareMathOperator{\Det}{det}      % Determinante
\DeclareMathOperator{\atan}{atan}    % tan^-1
\DeclareMathOperator{\Cov}{Cov}      % Kovarianz

% Interne Referenzen z.B. für definierte Begriffe
\newcommand{\link}[2]{\hyperref[#1]{#2}}
% Verweis auf weitere Informationen (Beweise, Erklärungen etc.)
\newcommand{\more}[1]{\,(vgl.\,\cite{#1})}
% Definierter Begriff in einer Definition
\newcommand{\defw}{\textbf}
% Ungenaue/Fragliche/Unbewiesene/Unbegründete Aussagen
\newcommand{\warn}[1]{(\textbf{#1})}

\title{%
\huge\textbf{Mathematisch-Stochastische Modelle: Markov-Ketten und Monte-Carlo-Simulationen} \\
[2em]\large Skript nach einer Vorlesung an der HTW Dresden\thanks{Dieses Skript
wurde von verschiedenen Studierenden erarbeitet, da es kein offiziell
bereitgestelltes Skript gibt. Die Latex-Quellen sind auf
\href{https://github.com/burrscurr/msm}{Github} verfügbar.}
}


\begin{document}
  \maketitle

  \tableofcontents

  \chapter{Grundlagen der Wahrscheinlichkeitsrechnung}
    \input{wahrscheinlichkeitstheorie.tex}
    \section{Zufallsvariablen}
    \input{zufallsvariablen-zufallsvektoren.tex}
    \section{Erwartungswert und Varianz}
    \input{erwartungswert-varianz.tex}

  \chapter{Wahrscheinlichkeitsverteilungen}
    \section{Diskrete Verteilungen}

\begin{definition}{Verteilungsfunktion}{vertf-disk}
Sei $X$ eine diskrete \link{def:zvar}{Zufallsvariable} mit Zustandsraum $x_0,
x_1, \ldots$. Die Funktion
\[
F_X:\R\to\R,\ F_X(z) = P(X \le z)
\]
heißt \defw{Verteilungsfunktion} der Zufallsvariable $X$. Die
Verteilungsfunktion eine Treppenfunktion, die an den Stellen $x_k$ um
$p_k = P(X=x_k)$ springt. Darum gilt:
\[
F_X(z) = \sum_{x_k\le z}P(X=x_k) = \sum_{x_k\le z} p_k
\]
\end{definition}

Eine Art Umkehrfunktion der Verteilungsfunktion ist die Quantilfunktion:
\begin{definition}{Quantilfunktion}{quantilf}
Sei $F_X(z) = P(X\le z)$ die Verteilungsfunktion einer diskreten Zufallsvariable
$X$. Dann heißt
\[
F_X^{-1}(z) = \mathrm{min}\{x\in\R: F(x) \ge z\},\quad z\in(0,1)
\]
die \defw{Quantilfunktion} von $X$.
\end{definition}

\medskip
Im Folgenden Abschnitt sei $X$ eine diskrete \link{def:zvar}{Zufallsvariable}
mit Zustandsraum $S$, $A \in S$ und $P$ eine
\link{def:verteilung}{Wahrscheinlichkeitsverteilung} dieser Zufallsvariable.


\subsection{Gleichverteilung}

Die Gleichverteilung ist eine sehr einfache Verteilung, bei der jeder Wert der
Zufallsvariable mit der gleichen Wahrscheinlichkeit auftritt:
\[
P(A) = \frac{|A|}{|S|}
\]

\subsection{Bernoulli-Verteilung ($B(p)$)}

Die Bernoulli-Verteilung beschreibt eine Zufallsvariable, die nur zwei mögliche
Zustände besitzt, hier bezeichnet als $S = \{0,1\}$. Der beliebig, aber fest
gewählte Ausgang $A$ besitzt die Wahrscheinlichkeit $0 \le p \le 1$, sodass
gilt:
\begin{align*}
P(X=1)&=p  \\
P(X=0)&=1-p
\end{align*}
Für Erwartungswert und Varianz gilt:
\begin{align*}
\E(X) &= p \\
\Var(X) &= p\cdot(1-p)
\end{align*}

\subsection{Binomialverteilung ($B(n,p)$)}

Die Binomialverteilung beschreibt die $n$-fache Durchführung eines Experiments
mit nur zwei komplementären Ausgängen werden. Die Zufallsvariable $X$ gibt an,
wie oft bei $n$-facher Wiederholung der beliebig, aber fest gewählte Ausgang $A$
eintritt. Damit kann $X$ die Werte $0, ..., n$ annehmen. Die Wahrscheinlichkeit
$p$ des Eintretens des gewählten Ausgangs bleibt dabei über alle $n$
Wiederholungen gleich. Es gilt:
\begin{align*}
P(X=k) &= \binom{n}{k}\cdot p^k\cdot(1-p)^{n-k} \\
\E(X) &= n \cdot p \\
\Var(X) &= n \cdot p \cdot (1-p) \\
\end{align*}

\subsection{Geometrische Verteilung ($Geo(p)$)}

Eine geometrische Verteilung entsteht durch die Wiederholung eines
Wahrscheinlichkeitsexperiments mit zwei komplementären Ausgängen. Die
Zufallsvariable $X$ beschreibt die Anzahl an Versuchen, die durchgeführt werden
müssen, \textbf{bevor} der beliebig, aber fest gewählte Ausgang $B$ eintritt.

Der Zustandsraum von $X$ ist damit $\N_0$. Sei $p$ die Wahrscheinlichkeit, dass Ausgang $B$ eintritt. 
Die Wahrscheinlichkeit $k$ Wiederholungen durchzuführen \textbf{bevor} $B$ das erste mal auftritt, ist:
\[
P(X=k) = p \cdot (1-p)^k
\]
Für Erwartungswert und Varianz gelten:
\begin{align*}
\E(X) &= \frac{1-p}{p} \\
\Var(X) &= \frac{1-p}{p^2}
\end{align*}


Die Zufallsvariable $Y = X+1$ beschreibt die Anzahl an Versuchen für das erste Eintreten von $B$.
Die Wahrscheinlichkeit, dass $B$ nach $k$ Wiederholungen eintritt, ist:
\[
P(Y=k) = p \cdot (1-p)^{k-1}
\]
Für Erwartungswert und Varianz gilt:
\begin{align*}
\E(Y) &= \frac{1}{p} \\
\Var(Y) &= \frac{1-p}{p^2}
\end{align*}


\subsection{Poisson-Verteilung ($Poi(\lambda)$)}

Die Poisson-Verteilung entsteht bei Vorgängen, die im Durchschnitt mit
konstanter Rate $\lambda \in (0, \infty)$ (Einheit: $\frac{1}{\text{Zeiteinheit}}$) 
in einem beliebigen, aber festen Zeitintervall auftreten. 
Die Zufallsvariable $X$ beschreibt, wie viele Vorgänge tatsächlich in 
dem Zeitintervall aufgetreten sind. Es gilt:
\begin{align*}
P(X=k) &= \frac{\lambda^k}{k!}\cdot\e^{-\lambda} \\
\E(X) &= \lambda \\
\Var(X) &= \lambda
\end{align*}

\section{Stetige Verteilungen}

\begin{definition}{Wahrscheinlichkeitsdichte}{dichte}
Sei $X$ eine stetige \link{def:zvar}{Zufallsvariable} mit Zustandsraum $S$.
Eine Funktion $\rho: \R \rightarrow \R$ heißt \defw{Wahrscheinlichkeitsdichte},
wenn gilt:
\begin{align*}
  \forall x: \rho(x) \ge 0 \\
  \int \rho(x) \,\mathrm{d}x = 1
\end{align*}
\end{definition}

Die Wahrscheinlichkeitsdichte kann verwendet werden, um die Wahrscheinlichkeit,
dass X bestimmte Werte annimmt, zu berechnen ($a,b\in\R$):
\begin{align*}
  P(X < b) &= \int_{-\infty}^{b}\rho(x)\,\mathrm{d}x\\
  P(a<X<b) &= \int_{a}^{b}\rho(x)\,\mathrm{d}x\\
  P(X < b) &= \int^{\infty}_{b}\rho(x)\,\mathrm{d}x
\end{align*}

\begin{definition}{Verteilungsfunktion}{vertf}
Sei $X$ eine \link{def:zvar}{Zufallsvariable}. Die Funktion
\[F_X:\R\to\R,\ F_X(z) = P(X \le z)\]
heißt \defw{Verteilungsfunktion} von $X$.
\end{definition}

Die Verteilungsfunktion ist monoton wachsend und rechtsstetig. Weiterhin gilt:
\begin{align*}
0\le F_X(z&)\le 1 \\
F(-\infty&) = 0 \\
F(\infty&) = 1
\end{align*}

Die Verteilungsfunktion $F_X$ kann wie folgt verwendet werden, um
Wahrscheinlichkeiten bezüglich der Zufallsvariable $X$ zu berechnen:
\begin{align*}
P(X>a) = &1 - F_X(a) \\
P(X\le b) = &F_X(b) \\
P(a < X \le b) = &F_X(b) - F_X(a)
\end{align*}

Im Folgenden Abschnitt sei $X$ eine stetige \link{def:zvar}{Zufallsvariable}
mit Zustandsraum $S$, $A \in S$ und $P$ eine
\link{def:verteilung}{Wahrscheinlichkeitsverteilung} dieser Zufallsvariable.


\subsection{Gleichverteilung ($U(a,b)$)}
\label{vert-gleich}

Die gleichverteilte Zufallsvariable $X$ nimmt die Werte $S=(a,b)$ an. Für die
Wahrscheinlichkeitsdichte gilt:
\[
\rho(x) = \frac{1}{b-a}\cdot\mathbb{I}_{(a,b)}(x)
\]
Dabei bezeichnet $\mathbb{I}_{(a,b)}$ die \defw{Indikatorfunktion}, die Werte im
Intervall von $(a,b)$ auf $1$ und alle anderen Werte auf $0$ abbildet.

\subsection{Exponentialverteilung ($Exp(\lambda)$)}
\label{vert-exp}

Die Exponentialverteilung wird vorrangig verwendet um die Dauer zufälliger 
Zeitintervalle zu modellieren. (Lebensdauer, Zeit zwischen 2 Ereignissen)
Für eine Exponentialverteilung mit konstanter Ereignisrate 
$\lambda > 0$ (Einheit: $\frac{1}{\text{Zeiteinheit}}$) gilt:
\begin{align*}
\rho(x) &= \lambda\cdot \e^{-\lambda x}\cdot\mathbb{I}_{(0, \infty)} \\
F(z) &= 1 - \e^{-\lambda\cdot z} \\
\E(X) &= \frac{1}{\lambda} \\
\Var(X) &= \frac{1}{\lambda^2}
\end{align*}

\subsection{Normalverteilung ($N(\mu, \sigma^2)$)}

In der Natur kommen Normalverteilungen vor wenn sich eine große Anzahl
unabhängiger Verteilungen überlagern. Für die Wahrscheinlichkeitsdichte gilt:
\begin{align*}
\rho(x) &= \frac{1}{\sqrt{2\pi\sigma^2}}\cdot exp(-\frac{(x-\mu)^2}{2\sigma^2}) \\
\E(X) &= \mu \\
\Var(X) &= \sigma^2
\end{align*}

\subsection{Standardnormalverteilung ($N(0,1)$)}
\label{vert-stdnormal}

Eine standardnormalverteilte Zufallsvariable nimmt im Mittel den Wert $0$ mit
einer Varianz von $1$ an. Es gilt:
\[
\rho(x) = \frac{1}{\sqrt{2\pi}}\cdot exp(-\frac{x^2}{2})
\]

Die \link{def:vertf}{Verteilungsfunktion} der Standardnormalverteilung wird mit $\Phi(z)$ bezeichnet.

Ist $X$ normalverteilt mit $\mu$ und $\sigma^2$, dann ist die Zufallsvariable
\[
Z=\frac{X-\mu}{\sigma}
\]
standardnormalverteilt.

Im Umkehrschluss kann eine standardnormalverteilte Zufallvariable beliebig transformiert werden:
\[
X = Z \cdot \sigma + \mu
\]


  \chapter{Generierung von Zufallszahlen}
    \input{zufallszahlengenerierung.tex}

  \chapter{Monte-Carlo}

Monte-Carlo Experimente bezeichnen eine Klasse von Algorithmen, 
welche durch wiederholtes generieren von Zufallszahlen ein Problem numerisch approximieren.
Diese Algorithmen können unter anderem genutzt werden, um besonders schwierige Probleme zu lösen,
welche auf analytischem Wege unmöglich wären.

Ein Beispiel einer Monte-Carlo Methode wurde in dieser Vorlesung bereits für die \link{algo:av-methode}{Annahme-Verwerfungs-Methode} eingesetzt.
In diesem Fall konnte durch das zufällige Generieren von Punkten die potenziell schwierige \link{algo:inv-methode}{Inversionsmethode} umgangen werden.
\\
\begin{wrapfigure}{r}{0.5\textwidth}
\centering
\includegraphics[width=0.4\textwidth]{mc-integration}
\end{wrapfigure}

Solche Simulationen sind anhand eines Beispiels leicht vorstellbar.
Es gibt eine quadratische Fläche, in welcher sich ein runder Pool befindet.
Wenn nun an zufälligen Stellen Golfbälle fallen gelassen werden, so landen diese entweder innerhalb oder außerhalb des Pools.
Wird nun die Menge der im Pool gelandeten Bälle mit der außerhalb gelandeten ins Verhältnis gesetzt,
so kann der Flächeninhalt des Pools ermittelt werden.
Dieser Algorithmus kann für Objekte beliebiger Komplexität und Dimensionalität angewendet werden.

\section{Numerische Integration}

Gegeben sei eine beliebige stückweise stetige Funktion:

\[f(x) \in (0, \infty) \qquad x \in [a,b]\]

Gesucht ist der Flächeninhalt von $f(x)$ im Intervall $[a,b]$.
Hierfür wird zuerst ein Rechteck mit Breite $b-a$ und Höhe $K = \max(f(x))$ bestimmt.
Nun kann das Verhältnis zwischen Rechteck und der Fläche unter dem Funktionsgraphen folgendermaßen als Wahrscheinlichkeit beschrieben werden:

\[p = \frac{I}{K(b-a)}\]

wobei $I$ das Integral der Funktion ist.
Wird diese Gleichung nach $I$ umgestellt und die Variablen $I,p$ mit ihren approximierten Variablen $\hat{I},\hat{p}$ ersetzt,
so entsteht die allgemeine Formel für den approximierten Flächeninhalt:

\[\hat{I} = K(b-a)\hat{p}\]

\subsection{Algorithmus A}

Diese Gleichung kann sehr einfach, 
aufbauend auf der \link{algo:av-methode}{Annahme-Verwerfungs-Methode},
angewendet werden.
Zuerst werden $N$ Iterationen der AVM durchgeführt und die Menge generierter Zahlen $A$ ermittelt.
Nun können $\hat{p}$, sowie $\hat{I}$ als: 

\[\hat{p} = \frac{A}{N}\]
\[\hat{I} = K(b-a)\hat{p}\]

bestimmt werden.

\subsection{Algorithmus B}

Der nächste Algorithmus wird folgendermaßen hergeleitet:

\[I = \int_a^b f(x)dx = 
\int_{-\infty}^\infty f(x)(b-a)
\frac{1}{(b-a)}\mathds{1}_{[a,b]}(x)\]

\[g(x) := f(x)(b-a)\]

\[\hat{I} = \frac{g(U_1) + ... + g(U_n)}{n}\]

Hierdurch entsteht die folgende nützliche Gleichung:

\[\hat{I} = (b-a)\frac{(f(U_1) + ... + f(U_n))}{n}\]

Diese kann intuitiv als \textbf{durchschnittlicher Wert von f(x)} multipliziert mit \textbf{Breite des Intervalls} bezeichnet werden.
Die Funktion wird also mithilfe des arithmetischen Mittels durch eine Gleichverteilung angenähert und dessen Integral als Rechteck berechnet.



\begin{algorithm}[h!]
    \DontPrintSemicolon
    \LinesNumbered
    
    Setze $A=0$\;
    \For(){k=1; k<N}{
      Erzeuge $x\sim U(a,b)$\;
      Setze $A=A+f(x)$\;
    }
    Setze $\hat{I} = \frac{(b-a)}{N}A$
    
    
    \caption{Algorithmus B}\label{algo:algo-b}
\end{algorithm}


% #############################################

% Ich bin mir ziemlich sicher Algo C hab ich hier falsch beschrieben.
% Ist zum Glück nicht Prüfungsrelevant

% #############################################

% \subsection{Algorithmus C}

% Der letzte Algorithmus orientiert sich an der Methode des \link{algo:imp-samp}{Importance Sampling}.
% Hier wird wieder eine Hilfsfunktion genutzt, welche es ermöglicht leicht (Inversionsmethode) Zufallszahlen zu generieren.
% Diese Zahlen werden dann anstelle der Uniformverteilung aus Algorithmus B genutzt um gezielter Punkte zu ermitteln.
% Um die dadurch veränderte Verteilung der Punkte muss zum Schluss nur noch ausgeglichen werden, 
% indem der Parameter $A$ um das Verhältnis der Funktion $f(x)$ und der Hilfsfunktion $\delta(x)$ inkrementiert wird. 
% Dieses Vorgehen ist im folgenden Algorithmus leicht zu erkennen:

% \begin{algorithm}[h!]
%   \DontPrintSemicolon
%   \LinesNumbered
  
%   Setze $A=0$\;
%   \For(){k=1; k<N}{
%     Erzeuge $x\sim \delta_x$\;
%     Setze $A=A+\frac{f(x)}{\delta_x(x)}$\;
%   }
%   Setze $\hat{I} = \frac{(b-a)}{N}A$
  
  
%   \caption{Algorithmus C}\label{algo:algo-c}
% \end{algorithm}

\section{Das Gesetz der großen Zahlen}

Das GGZ besagt, dass ein durch wiederholte Zufallsexperimente generierter Wert ebenfalls eine Zufallsvariable ist.
Wenn also eine Zufallsvariable $X$ mit $\E(X)=\mu_x$ und $\Var(X)=\sigma_x^2$ gegeben ist, 
können Zahlen $x_1 \dots x_n \sim X$ i.i.d. gezogen werden. 
Beispielsweise ist die Summe $S_n$ dieser Zahlen, und auch deren arithmetisches Mittel $\overline{X}_n$ ebenfalls eine Zufallsvariable.
\begin{align*}
 S_n &= \sum_{i=1}^n x_i \\
\overline{X}_n &= \frac{1}{n}\sum^n_{i=1}x_i
\end{align*}

Diese neuen Zufallsvariablen haben einen eigenen Erwartungswert und Varianz:
\begin{align*}
\E(S_n) &= n \cdot \mu_x \\
\Var(S_n) &= n \cdot \sigma_x^2 \\
\\
\E(\overline{X}_n) &= \mu_x \\
\Var(\overline{X}_n) &= \frac{\sigma_x^2}{n}
\end{align*}


Der Erwartungswert entspricht also dem der ursprünglichen Verteilung und die Varianz sinkt mit steigender Zahl von Zufallsexperimenten.
Das bedeutet insbesondere, dass der Erwartungswert einer Zufallsvariable durch das arithmetische Mittel von ebenso verteilten Zufallszahlen angenähert werden kann, wenn $n$ nur groß genug ist.

\section{Der zentrale Grenzwertsatz}

Der zentrale Grenzwertsatz besagt, dass der Erwartungswert einer Zufallszahl für große $n$ Normalverteilt:

\[\overline{X}_n \approx N(\mu, \frac{\sigma^2}{n})\]
bzw.
\[ \frac{\overline{X}_n -\mu_x}{\sigma_x} \cdot \sqrt{n} \approx N(0,1) \]

Abweichungen des Erwartungswertes können also mit den gleichen Methoden bestimmt und geschätzt werden, wie bei der Normalverteilung.
Bei solchen Problemstellungen existieren drei Variablen, welche die Genauigkeit beschreiben:

\begin{align*}
  \alpha&\: ...\: \text{Risikoschwelle}\\
  n&\: ...\: \text{Zahl der Wiederholungen}\\
  \epsilon&\: ...\: \text{Genauigkeit}\\
\end{align*}

Wenn in Aufgabenstellungen zwei dieser gegeben sind, ist es möglich die fehlende Variable zu bestimmen:

\begin{align*}
  \text{Geg.:} \:  &n, \epsilon\\
  \alpha &\ge 2-2\phi\left(\frac{\epsilon\sqrt{n}}{\sigma}\right)\\
  \text{Geg.:} \:  &\alpha, \epsilon\\
  n &\ge \left(Z_{1-\frac{\alpha}{2}}\frac{\sigma}{\epsilon}\right)^2\\
  \text{Geg.:} \:  &n, \alpha\\
  \epsilon &\ge Z_{1-\frac{\alpha}{2}}\frac{\sigma}{\sqrt{n}}\\
\end{align*}

Hierbei ist $Z_{1-\frac{\alpha}{2}}$ das Quantil der Standardnormalverteilung mit dem Wert $1-\frac{\alpha}{2}$.

\section{Der Satz von Moivre-Laplace}

Ist $Y$ Binominalverteilt $Y \sim B(n,p)$ und sind $np$ und $n(1-p)$ groß, so kann die Binomialverteilung durch eine Normalverteilung angenähert werden.
Die Parameter sind dann:

\[\mu = np \qquad \sigma^2=np(1-p)\]

Außerdem muss bei der Rechnung mit dieser Normalverteilung noch eine Stetigkeitskorrektur von $0.5$ eingefügt werden:

\[P(a < Y \le b) \approx 
  \phi\left(\frac{b+\mathbf{0.5}-np}{\sqrt{np(1-p)}}\right) - 
  \phi\left(\frac{a+\mathbf{0.5}-np}{\sqrt{np(1-p)}}\right)\]

Die Faustregel für die Gültigkeit dieser Regel lautet:

\[np(1-p) \ge 9\]

% VL 08 beinhaltet den gleichen Stoff wie 7 + Anwendung und Beispiele


  \chapter{Markov-Ketten}
    \input{markov-eigenschaften.tex}
    \section{Langzeitverhalten}
    \newcommand{\stat}{\underline{\pi}} % stationäre Wahrscheinlichkeit

Dieser Abschnitt beschäftigt sich mit der \link{def:mk-vert}{Verteilung der
Markovkette} nach unendlich vielen Schritten.

\textbf{Achtung: Die hier verwendete
Notation weicht zum Teil deutlich von den Vorlesungsvideos ab.}

\begin{definition}{Stationäre Verteilung}{mk-stat}
Sei $(X)_{n\in N_0}$ eine Markovkette mit \link{def:mk-matr}{Übergangsmatrix}
$\Pi$ und endlichem Zustandsraum $S$. Dann heißt die
\link{def:mk-vert}{Verteilung} $\stat$ \defw{stationäre Verteilung},
wenn gilt:
\[
\stat\cdot\Pi = \stat
\]
\end{definition}

Gelangt oder startet man in einer stationären Verteilung, ändert sich die
Wahrscheinlichkeit, sich in einem Zustand zu befinden, nicht mehr.

\subsection{Ergodische Markovketten}

\begin{theorem}{Hauptsatz für ergodische Markovketten}{ergod}
Sei $(X)_{n\in N_0}$ eine \link{def:mk-ergod}{ergodische} Markovkette mit
\link{def:mk-matr}{Übergangsmatrix} $\Pi$ und $y$ ein Zustand der Markovkette
mit \link{def:mk-rueckk}{Rückkehrzeit} $T_y$. Dann
besitzt die Markovkette \emph{genau} eine stationäre Verteilung $\stat$
für die gilt:
\[
\lim_{n\to\infty}\Pi(X_n = y) = lim_{n\to\infty}\Pi^n(x, y) = \stat(y)
\]
Weiterhin existiert ein Zusammenhang zwischen der erwarteten Rückkehrzeit und
der stationären Verteilung:
\[
\stat(y) = \frac{1}{\E(T_y|X_0=y)}
\]
\end{theorem}

Je weiter eine ergodische Markovkette läuft, desto mehr nähert sich die
Wahrscheinlichkeitsverteilung an die stationäre Verteilung an. Die stationäre
Verteilung gibt – zumindest auf lange Sicht – an, mit welcher Wahrscheinlichkeit
man sich in den jeweiligen Zuständen befindet.

Die stationäre Verteilung einer ergodischen Markovkette kann berechnet
werden. Dafür nutzt man die Tatsache, dass die stationäre Verteilung (per
Definition) ein Eigenvektor\more{mfnf-ew-ev} der Übergangsmatrix. Zusätzlich
muss wird eine Gleichung eingefügt, die die Normierung der stationären
Verteilung auf 1 sicherstellt. Es ergibt sich folgendes
lineares Gleichungssystem:
\begin{equation}\label{eq:stat-lgs}
\begin{pmatrix}
\Pi_{00}-1 & \Pi_{10} & \ldots & \Pi_{j0} \\
\Pi_{01} & \Pi_{11}-1 &        & \Pi_{j1} \\
\vdots   &            & \ddots & \vdots \\
\Pi_{0i} & \Pi_{1i} & \ldots &   \Pi_{ji}-1 \\
    1    &     1    & \ldots &   1
\end{pmatrix}\cdot \vec{\stat} = \begin{pmatrix}0\\0\\\vdots\\0\\1\end{pmatrix}
\end{equation}
Wichtig ist auch, dass die Matrix $\Pi$ transponiert aufgeschrieben, da eine
Spalte der Matrix einer Gleichung entspricht.

\subsection{Nichtergodische Markovketten}

\newcommand{\lzv}{\Pi^\infty} % Langzeitverhalten

Das Langzeitverhalten von nichtergodischen Markovketten hängt von Startzustand
ab. Es wird in der Matrix $\lzv$ zusammengefasst, die jedem Startzustand die auf
lange Sicht zu erwartende Verteilung zuordnet. Die Regeln zur Bestimmung des
Langzeitverhaltens unterscheiden sich je nach dem ob der Startzustand in
einer \link{def:mk-abg}{abgeschlossenen} oder \link{def:mk-trans}{transienten}
Klasse liegt.

\medskip
Im Folgenden sei $(X)_{n\in N_0}$ eine Markovkette mit endlichem Zustandsram $S$
und \link{def:mk-matr}{Übergangsmatrix} $\Pi$.

Das Langzeitverhalten bei Start in einem Zustand einer
\link{def:mk-abg}{abgeschlossenen} Klasse $C$ entspricht der
\link{def:mk-stat}{stationäre Verteilung} $\stat^{(C)}$ der auf die Zustände in
$C$ beschränkten Markovkette (siehe \eqref{eq:stat-lgs}). Der Übergang in einen
Zustand außerhalb der Klasse $C$ ist ausgeschlossen. Für $x, x' \in C$ und
$y\notin C$ gilt:
\begin{align}\label{eq:lzv-abg-abg}
  \lzv(x,x') &= \stat^{(C)}(x') \\
  \label{eq:lzv-abg-x}
  \lzv(x,y) &= 0
\end{align}

\medskip

Da \link{def:mk-trans}{transiente} Klassen nur verlassen, aber nicht in sie
zurückgekehrt werden kann, ist die langfristige Wahrscheinlichkeit, sich in
einem transienten Zustand zu befinden, null. Die Wahrscheinlichkeit, bei Start
in einem transienten Zustand $t$ langfristig in die abgeschlossene Klasse $C$
überzugehen, wird mit $p_{t\to C}$ bezeichnet \warn{andere Schreibweise als in
der Vorlesung}. Diese sogenannte Absorptionswahrscheinlichkeit verteilt sich
dann innerhalb von $C$ gemäß der berechneten stationären Verteilung.

Seien $x,x'$ transiente Zustände und $y$ Zustand einer
abgeschlossenen Klasse $C$. Dann gilt:
\begin{align}
  \label{eq:lzv-trans-abg}
  \lzv(x,y) &= p_{t\to C}\cdot\stat^{(C)}(y) \\
  \label{eq:lzv-trans-trans}
  \lzv(x,x') &= 0
\end{align}

Die Absorptionswahrscheinlichkeiten von Zuständen einer transienter Klasse $T$
bezüglich des Übergangs in die abgeschlossene Klasse $C$ bilden ein
lineares Gleichungssystem. Jede Gleichung für $t\in T$ ergibt sich
aus der Wahrscheinlichkeit des direkten Überangs in $C$ und der
Wahrscheinlichkeit, indirekt über die anderen Zustände von $T$ in $C$
überzugehen:
\[
p_{t\to C} = \sum_{c\in C} \Pi(t,c) + \sum_{t'\in T}\Pi(t,t')\cdot p_{t'\to C}
\]

\begin{example}{Langzeitverhalten einer nichtergodischen Markovkette}{mk-lzv-nerg}
Gegeben sei folgende Markovkette:

\begin{minipage}{0.45\textwidth}
  \begin{tikzpicture}
    \node[v] (1) at (0,1.5) {1};
    \node[v] (2) at (-1.5,0) {2};
    \node[v] (3) at (0,0) {3};
    \node[v] (4) at (1.5,0) {4};

    \draw[e] (1) to[loop above] ();
    \draw[e] (2) to[loop below] ();
    \draw[e] (3) to[loop below] ();
    \draw[e] (4) to[loop below] ();
    \draw[e] (1) -- (2);
    \draw[e] (1) -- (3);
    \draw[e] (1) -- (4);
    \draw[e] (2) to[bend left] (3);
    \draw[e] (3) to[bend left] (2);
  \end{tikzpicture}
\end{minipage}
\begin{minipage}{0.45\textwidth}
  \[\Pi = \begin{pmatrix}
    0.1 & 0.4 & 0.3 & 0.2 \\
     0  & 0.2 & 0.8 &  0  \\
     0  & 0.9 & 0.1 &  0  \\
     0  &  0  &  0  &  1
  \end{pmatrix}\]
\end{minipage}

Die Markovkette ist nicht \link{def:mk-irred}{irreduzibel}, da mehr als eine
Klasse existiert ($S_{/\lr} = {S_1, S_{2,3}, S_{4}}$), Satz \ref{satz:ergod}
kann also nicht angewandt werden.

\medskip
Da Klasse $S_4$ abgeschlossen und Zustand 4 absorbierend ist, ergibt sich als
Sonderfall von \eqref{eq:lzv-abg-abg} und \eqref{eq:lzv-abg-x} folgende
Verteilung:
\[\begin{pmatrix}0 & 0 & 0 & 1\end{pmatrix}\]

Klasse $S_{2,3}$ ist abgeschlossen. Die stationäre Verteilung $\stat^{(S_{2,3})}$
ergibt sich als Lösung des Gleichungssystems (Vergleiche \eqref{eq:stat-lgs})
und bestimmt $\lzv(2..3, 2..3)$:
\[
\begin{pmatrix}
0.2 -1 & 0.9     \\
0.8    & 0.1 - 1 \\
   1   &  1
\end{pmatrix}\cdot \vec{\stat}^{(S_{2,3})} = \begin{pmatrix}0\\0\\1\end{pmatrix}
\iff \vec{\stat}^{(S_{2,3})} =
\renewcommand\arraystretch{1.3}
\begin{pmatrix}0.53\\0.47\end{pmatrix}
\]

Zustand 1 ist transient, damit gilt $\lzv(1,1) = 0$
(\eqref{eq:lzv-trans-trans}). Als Zwischenergebnis bestimmen wir die
Absorptionswahrscheinlichkeiten $p_{abs,1}(S_{2,3})$ und $p_{abs,1}(S_{4})$:
\begin{align*}
p_{abs,1}(S_{2,3}) &=&  0.4 + 0.3 + 0.1 \cdot p_{abs,1}(S_{2,3}) & \implies
p_{abs,1}(S_{2,3}) = \frac{0.7}{0.9} = 0.78  \\
p_{abs,1}(S_{4})   &=&  0.2 + 0.1 \cdot p_{abs,1}(S_4) &\implies
p_{abs,1}(S_4) =\frac{0.2}{0.9} = 0.22
\end{align*}
Durch Anwendung von Gleichung \eqref{eq:lzv-trans-abg} ergibt sich:
\begin{align*}
\lzv(1,2) &=& p_{abs,1}(S_{2,3}) &\cdot\stat^{(S_{2,3})}(2) &= 0.41 \\
\lzv(1,3) &=& p_{abs,1}(S_{2,3}) &\cdot\stat^{(S_{2,3})}(3) &= 0.37 \\
\lzv(1,4) &=& p_{abs,1}(S_4)     &\cdot\stat^{(S_4)}    (4) &= 0.22
\end{align*}
Damit sind alle Einträge der Matrix $\lzv$ bestimmt:
\[
\lzv = \begin{pmatrix}
   0  & 0.41 & 0.37 & 0.22 \\
   0  & 0.53 & 0.47 &  0   \\
   0  & 0.53 & 0.47 &  0   \\
   0  &  0   &  0   &  1
\end{pmatrix}
\]
Als Probe kann sichergestellt werden, dass die Matrix stochastisch ist, sich
also die Einträge jeder Zeile zu 1 summieren.
\end{example}

\subsection{Reversibilität}

\begin{definition}{Reversibilität; Detaillierte Balance}{mk-rev}
Sei $(X)_{n\in N_0}$ eine Markovkette mit Zustandsraum $S$, Übergangsmatrix
$\Pi$ und $\pi$ eine \link{def:mk-vert}{Verteilung}. Die Markovkette heißt
\defw{reversibel}, wenn gilt:
\[
\forall x,y\in S:\ \pi(y)\cdot\Pi(y,x) = \pi(x)\cdot\Pi(x,y)
\]
Diese Bedingung wird auch als \defw{detaillierte Balance} bzw. als
\defw{detailliertes Gleichgewicht} bezeichnet.
\end{definition}

In einer reversiblen Markovkette kann also nicht unterschieden werden, ob der
Prozess vorwärts oder rückwärts abläuft.

\begin{theorem}{}{mk-db}
Sei $\pi$ eine Verteilung, die die Bedingung der detaillierten Balance
für eine Markovkette erfüllt. Dann ist $\pi$ eine
\link{def:mk-stat}{stationäre Verteilung}.
\end{theorem}

Reversibilität ist eine stärkere Bedinung als Stationarität, es gibt also
stationäre Verteilungen, die nicht die Bedingung der detaillierten Balance
erfüllen.

\begin{theorem}{Schnittprinzip}{mk-schnittp}
Sei $(X)_{n\in N_0}$ eine \link{def:mk-ergod}{ergodische} Markovkette. Ist der
\link{def:mk-igraph}{Interaktionsgraph} von $(X)$ linear, das jeder Schnitt
zwischen zwei verbundenen Zuständen zerlegt den Graph in zwei disjunkte und
unverbundene Teile, so ist die stationäre Verteilung immer reversibel.
\end{theorem}

Da wir wissen, dass eine ergodische Markovkette immer genau eine stationäre
Verteilung besitzt (Satz \ref{satz:ergod}), folgt aus dem Schnittprinzip, dass
diese Verteilung auch reversibel ist. Dann können wir für die Verteilung die
detaillierten Balance annehmen. Das ist besonders hilfreich, wenn der
betrachtete Prozess sehr viele Zustände besitzt und die stationäre Verteilung
aufwändig zu berechnen ist.

\begin{example}{Warteschlange}{}
Ein bestimmter Server kann bis zu 100 Anfragen parallel bearbeiten. Pro Sekunde
erhält der Server mit Wahrscheinlichkeit $p^+ = 0.2$ eine neue Anfrage und schließt
mit Wahrscheinlichkeit $p^-=0.21$ eine Anfrage ab. Werden bereits 100 Anfragen
bearbeitet, werden neue Anfragen abgewiesen.

\emph{Wie hoch ist die Wahrscheinlichkeit, dass ein ankommender Auftrag abgewiesen
wird? Wie hoch ist die Wahrscheinlichkeit, dass die Maschine leerläuft?}

Die Anzahl der aktiven Anfragen kann durch eine Markovkette mit Zustandsraum
$S=\{0, 1,\ldots, 100\}$ beschrieben werden:

\medskip
\begin{tikzpicture}
  \node[v] (0) at (0,1) {0};
  \node[v] (1) at (2,1) {1};
  \node[v] (2) at (4,1) {2};
  \node[v] (n) at (6, 1) {$\ldots$};
  \node[v] (99) at (8,1) {99};
  \node[v] (100) at (10,1) {100};

  \draw[e] (0) to[loop left] node[anchor=south] {$p^-$} ();
  \draw[e] (100) to[loop right] node[anchor=south] {$p^+$} ();
  \draw[e] (0) to[bend left] node[anchor=south] {$p^+$} (1);
  \draw[e] (1) to[bend left] node[anchor=north] {$p^-$} (0);
  \draw[e] (1) to[bend left] node[anchor=south] {$p^+$} (2);
  \draw[e] (2) to[bend left] node[anchor=north] {$p^-$} (1);
  \draw[e] (2) to[bend left] node[anchor=south] {$p^+$} (n);
  \draw[e] (n) to[bend left] node[anchor=north] {$p^-$} (2);
  \draw[e] (n) to[bend left] node[anchor=south] {$p^+$} (99);
  \draw[e] (99) to[bend left] node[anchor=north] {$p^-$} (n);
  \draw[e] (99) to[bend left] node[anchor=south] {$p^+$} (100);
  \draw[e] (100) to[bend left] node[anchor=north] {$p^-$} (99);
\end{tikzpicture}

Die Markovkette besteht nur aus einer einzigen Klasse, ist daher abgeschlossen,
rekurrent und irreduzibel. Da die Zustände 0 und 100 Periode 1 besitzten, ist
die Markovkette aperiodisch. Folglich ist die Markovkette auch ergodisch, es
existiert also eine eindeutige stationäre Verteilung $\stat$.

Diese Verteilung ist wegen der 101 Zustände sehr aufwändig zu berechnen. Wir
können jedoch das Schnittprinzip (Satz \ref{satz:mk-schnittp}) anwenden, da der
Interaktionsgraph der Markovkette linear ist. Das bedeutet, dass für die
stationäre Wahrscheinlichkeit die Bedingung der detaillierten Balance gilt:
\begin{align*}
\forall x,y\in S:\ &\pi(y)\cdot\Pi(y,x) = \pi(x)\cdot\Pi(x,y) \\
\implies&\stat(i)\cdot p^+ = \stat(i+1)\cdot p^- \\
\implies&\stat(i+1) = \frac{p^+}{p^-}\cdot\stat(i) \\
\implies&\stat = (c, c\cdot\gamma,c\cdot\gamma^2,\ldots,
c\cdot\gamma^{100})^{\T},\ \gamma = \frac{p^+}{p^-}, c\in\R
\end{align*}
Da $\stat$ eine Verteilung ist, wissen wir dass die Einträge insgesamt 1
ergeben. Damit kann $c$ bestimmt werden:
\begin{align*}
1 &= \sum_{i=0}^{100}c\cdot\gamma^i = c \sum_{i=0}^{100}\gamma^i \\
\overset{\text{Geom. Summenformel}}{\iff} 1 &= c\cdot\frac{1-\gamma^{101}}{1-\gamma} \\
\implies \stat(i) &= c\cdot\gamma^i = \frac{1-\gamma}{1-\gamma^{101}}\cdot \gamma^i
\end{align*}

Die Wahrscheinlichkeit, dass ein ankommender Auftrag abgewiesen wird, ist
$\stat(100)=3.6\cdot10^{-4}$, der Server also bereits voll ausgelastet ist. Die
Wahrscheinlichkeit für Leerlauf ist $\stat(0) = 4.79\cdot10^{-2}$.
\end{example}

    \section{Markovketten mit unendlichem Zustandsraum}
    \input{markov-unendlich.tex}

  \printbibliography
\end{document}
