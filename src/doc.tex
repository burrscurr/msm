\documentclass{report}

\usepackage[utf8]{inputenc}
\usepackage[T1]{fontenc}
\usepackage[ngerman]{babel}

\usepackage{hyphenat}
\hyphenation{Mathe-matik}

% Seitenlayout
\usepackage[a4paper,width=135mm,top=25mm,bottom=25mm,bindingoffset=0mm]{geometry}

% Bilder/Darstellungen
\usepackage{graphicx}
\graphicspath{{./img/}}
\usepackage{float}
\usepackage{wrapfig}

% Graphen
\usepackage{tikz}
\tikzset{
    v/.style={circle,minimum size=2ex, text width=5mm, align=center, draw=black},
    e/.style={-stealth, thick}}

% Mathematikfoo
\usepackage{amsmath}
\usepackage{amsthm}
\usepackage{amssymb}
\usepackage{dsfont}   % \mathds{1}

% Algorithmen
\usepackage[
german,
ruled
]{algorithm2e}

% Verlinkungen
\usepackage{hyperref}
\usepackage{cleveref}

% Literatur
\usepackage{biblatex}
\addbibresource{ref/zufallszahlen.bib}
\addbibresource{ref/markov.bib}

% Einrahmungen
\usepackage[most,many]{tcolorbox}

% Layout von https://tex.stackexchange.com/questions/369430/
% Definitionen
\tcbset{definitionstyle/.style={
    enhanced,
    sharp corners,
    attach boxed title to top left={
      xshift=-1mm,
      yshift=-4mm,
      yshifttext=-3mm
    },
    top=1.5ex,
    colback=white,
    colframe=green!40!black,
    fonttitle=\bfseries,
    boxed title style={
      sharp corners,
    size=small,
    colback=green!40!black,
    colframe=green!40!black
  }
}}
\newtcbtheorem{definition}{Definition}{definitionstyle}{def}

% Sätze
\tcbset{theoremstyle/.style={
    enhanced,
    sharp corners,
    attach boxed title to top left={
      xshift=-1mm,
      yshift=-4mm,
      yshifttext=-3mm
    },
    top=1.5ex,
    colback=white,
    colframe=blue!95!black,
    fonttitle=\bfseries,
    boxed title style={
      sharp corners,
    size=small,
    colback=blue!95!black,
    colframe=blue!95!black,
  }
}}

\newtcbtheorem{theorem}{Satz}{theoremstyle}{satz}

% Beispiele
\tcbset{examplestyle/.style={
    enhanced,
    sharp corners,
    attach boxed title to top left={
      xshift=-1mm,
      yshift=-4mm,
      yshifttext=-3mm
    },
    top=1.5ex,
    colback=white,
    colframe=white!55!black,
    fonttitle=\bfseries,
    boxed title style={
      sharp corners,
    size=small,
    colback=white!50!black,
    colframe=white!50!black,
  }
}}
\newtcbtheorem{example}{Beispiel}{examplestyle}{bsp}

% Lemmata
\newtheoremstyle{lemmastyle}
  {0.5}% space above
  {1}% space below
  {\itshape}% body font
  {}% indent
  {\textbf}%
  {}%
  { }% no newline after 'Lemma'
  {\thmname{#1}.}% don't print a lemma number
\theoremstyle{lemmastyle}
\newtheorem{lemma}{Lemma}


% Abkürzungen für Mathezeugs
\newcommand{\N}{\mathbb{N}}
\newcommand{\R}{\mathbb{R}}
\newcommand{\e}{\mathrm{e}}
\newcommand{\lr}{\leftrightarrow}
\newcommand{\pd}[2]{\frac{\partial #1}{\partial #2}} % partielle Ableitungen
\newcommand{\zvec}[1]{\overrightarrow{\mathrm{#1}}}
\newcommand{\T}{\top}                % Matrixtransposition
\DeclareMathOperator{\E}{E}          % Erwartungswert
\DeclareMathOperator{\Var}{Var}      % Varianz
\DeclareMathOperator{\Det}{det}      % Determinante
\DeclareMathOperator{\atan}{atan}    % tan^-1
\DeclareMathOperator{\Cov}{Cov}      % Kovarianz

% Interne Referenzen z.B. für definierte Begriffe
\newcommand{\link}[2]{\hyperref[#1]{#2}}
% Verweis auf weitere Informationen (Beweise, Erklärungen etc.)
\newcommand{\more}[1]{\,(vgl.\,\cite{#1})}
% Definierter Begriff in einer Definition
\newcommand{\defw}{\textbf}
% Ungenaue/Fragliche/Unbewiesene/Unbegründete Aussagen
\newcommand{\warn}[1]{(\textbf{#1})}

\title{%
\huge\textbf{Mathematisch-Stochastische Modelle: Markov-Ketten und Monte-Carlo-Simulationen} \\
[2em]\large Skript nach einer Vorlesung an der HTW Dresden\thanks{Es gibt kein
offizielles Skript (WTF?)}}

\author{Max Ziermann\thanks{Quellen sind auf \href{https://github.com/burrscurr/msm}
{Github} verfügbar. Mitarbeit ist ausdrücklich erwünscht!}}


\begin{document}
  \maketitle

  \tableofcontents

  \chapter{Grundlagen der Wahrscheinlichkeitsrechnung}
    \input{wahrscheinlichkeitstheorie.tex}
    \section{Zufallsvariablen}
    \begin{definition}{Zufallsvariable}{zvar}
Eine Funktion $X: \Omega \to \R$ wird als \defw{Zufallsvariable}
bezeichnet. Die Zufallsvariable ordnet jedem Ereignis einer
\link{def:ealg}{Ereignisalgebra} eine reelle Zahl zu.

Der Wertebereich der Zufallszahl wird als \defw{Zustandsraum} $S = X(\Omega)$
bezeichnet.

Ist $S$ endlich oder abzählbar unendlich, wird die Zufallsvariable als
\defw{diskret}, ist $S$ überabzählbar unendlich als \defw{stetig} bezeichnet.
\end{definition}

\begin{definition}{Wahrscheinlichkeitsverteilung}{verteilung}
Sei $A$ eine \link{def:ealg}{Ereignisalgebra}, $X$ eine \link{def:zvar}{Zufallsvariable}
mit Zustandsraum $S$. Dann heißt Funktion $P: S \rightarrow [0,1]$ definiert
durch
\[
P(A) = P(X^{-1}(A)), A \in S
\]
eine \defw{Wahrscheinlichkeitsverteilung}. Eine Verteilung einer stetigen
Zufallsvariable wird als \defw{stetige Verteilung}, die einer diskreten
Zufallsvariable als \defw{diskrete Verteilung} bezeichnet.
\end{definition}

\begin{definition}{Unabhängigkeit von Zufallsvariablen}{zvar-unabh}
Seien $X: \Omega \to \R$, $Y: \Omega \to \R$ \link{def:zvar}{Zufallsvariablen}. $X,
Y$ heißen \defw{unabhängig}, wenn für $a,b,c,d \in \R$, $a\le b, c\le d$ gilt:
\[
P(a < X\le b,c<Y\le d) = P(a<X\le b)\cdot P(c<Y\le d)
\]
\end{definition}

\section{Zufallsvektoren}

Zufallsvektoren sind Kombinationen von Zufallsvariablen. Grundsätzlich ist die
Anzahl der Zufallsvariablen beliebig; hier werden jedoch primär zweidimensionale
Zufallsvariablen betrachtet.

\begin{definition}{Zufallsvektor}{zvektor}
Seien $X_1, ..., X_n$ \link{def:zvar}{Zufallsvariablen}. Die Zusammenfassung
\[
\zvec{X} = (X_1, ..., X_n)^\T
\]
zu einem Vektor heißt \defw{Zufallsvektor}. Sind alle Komponenten des Vektors
diskret beziehungsweise stetig, heißt der Zufallsvektor diskret beziehungsweise
stetig.
\end{definition}

\subsection{Gemeinsame Verteilung}

Die Werte eines (zweidimensionalen) diskreten Zufallsvektors lassen sich in einer Matrix
zusammenfassen:

\begin{definition}{Gemeinsame Verteilung eines Zufallsvektors}{vert-zvektor}
Sei $\zvec{X} = (X, Y)^\T$ ein diskreter Zufallsvektor, wobei die
Zufallsvariable $X$ die Werte $x_0, x_1, ..., x_n$ und $Y$ die Werte $y_0, y_1, ...,
y_m$ annimmt. Dann bezeichnet die Matrix
\[
(p_{ij})_{i=0,...,n;j=0,...,m} \qquad mit\ p_{ij} = P(X=x_i, Y=y_j)
\]
die \defw{gemeinsame Verteilung} von $\zvec{X}$.

Die Verteilung eines stetigen Zufallsvektors $(X,Y)^\T$ wird durch die gemeinsame
Dichte $\rho_{X,Y}(x,y)$ beschrieben.
\end{definition}

\begin{definition}{Randverteilung}{randv}
Sei $(X,Y)^\T$ ein diskreter Zufallsvektor mit gemeinsamer Verteilung $p$.
Die Summierung von Zeilen bzw. Spalten der Matrix werden als
\defw{Randverteilung} bezeichnet:
\begin{align*}
p_{i.}:=P(X=x_i) = \sum_j p_{ij} \\
p_{.j}:=P(Y=y_j) = \sum_i p_{ij}
\end{align*}
\end{definition}

Analog gilt für stetige Zufallsvektoren:
\begin{definition}{Randdichte}{randd}
Sei $(X,Y)^\T$ ein stetiger Zufallsvektor mit gemeinsamer \link{def:dichte}{Dichte}
$\rho_{(X, Y)}$. Dann werden
\begin{align*}
\rho_X(x) = \int\rho_{(X,Y)}(x,y)\mathrm{d} y\quad x\in\R\\
\rho_Y(y) = \int\rho_{(X,Y)}(x,y)\mathrm{d} x\quad y\in\R
\end{align*}
als \defw{Randdichten} des Zufallsvektors bezeichnet.
\end{definition}

\begin{theorem}{}{randd}
Sei $(X,Y)^\T$ ein Zufallsvektor von unabhängigen Zufallsvariablen $X$ und $Y$
mit zugehörigen Randdichten $\rho_X$ und $\rho_Y$. Dann kann die gemeinsame
Verteilung $\rho_{X,Y}$ berechnet werden durch:
\[
\rho_{X,Y}(x,y) = \rho_X(x)\cdot\rho_Y(y)
\]
\end{theorem}

\subsection{Bedingte Wahrscheinlichkeit}

Analog zur \link{def:bedw}{bedingten Wahrscheinlichkeit} von Ereignissen lässt sich
auch für Zufallsvektoren eine bedingte Wahrscheinlichkeit definieren:

\begin{definition}{Bedingte Wahrscheinlichkeit}{bedwahr}
Sei $\zvec{X} = (X, Y)^\T$ ein diskreter Zufallsvektor mit
\link{def:vert-zvektor}{gemeinsamer Verteilung} $(p_{ij})_{i,j=0,1,...}$. Dann ist
mit
\[
P(Y=y_j|X=x_i) := \frac{P(X=x_i, Y=y_j)}{P(X=x_i)} = \frac{p_{ij}}{p_{i.}}
\]
die \defw{bedingte Wahrscheinlichkeit} von $Y=y_j$ unter Bedingung $X=x_i$ gegeben.
\end{definition}

\begin{definition}{Bedingte Dichte}{bedd}
Ist $\zvec{X}$ ein stetiger Zufallsvektor mit gemeinsamer Dichte
$\rho_{(X,Y)}$, so bezeichnen die Funktionen
\begin{align*}
\rho_{X|Y=y}(x) = \frac{\rho_{X,Y}(x,y)}{\rho_Y(y)}\\
\rho_{Y|X=x}(y) = \frac{\rho_{X,Y}(x,y)}{\rho_X(x)}
\end{align*}
die \defw{bedingte Dichte} von $X$ unter $Y=y$ bzw. $Y$ unter $X=x$.
\end{definition}

Ebenso analog zur \link{def:bedw}{bedingten Wahrscheinlichkeit} von Ereignissen kann
der \link{satz:bayes}{Satz von Bayes} für diskrete bzw. stetige Zufallsvektoren
formuliert werden:
\begin{align*}
p_{ij} = P(Y=y_j|X=x_i)\cdot p_{i.}\\
\rho_{X,Y} = \rho_{Y|X=x}(y)\cdot\rho_X(x)
\end{align*}

    \section{Erwartungswert und Varianz}
    \subsection{Erwartungswert}

Der Erwartungswert beschreibt die Zahl, die eine Zufallsvariable im Mittel
annimmt.

\begin{definition}{Erwartungswert}{ewert}
Sei $X$ eine \link{def:zvar}{Zufallsvariable} mit Zustandsraum $S=\{x_0, x_1,
...\}$ und Einzelwahrscheinlichkeiten $p_k=P(X=x_k)$. Dann heißt
\[
\E(X) =\langle X\rangle := \sum_k x_kp_k
\]
Erwartungswert von $X$. Ist $X$ eine stetige Zufallsvariable mit Dichte
$\rho_X$, gilt:
\[
\E(X) =\langle X\rangle := \int x\cdot\rho_X(x)\mathrm{d}x
\]
Oft wird $\mu$ als Zeichen für den Erwartungswert verwendet.
\end{definition}

\begin{theorem}{Rechenregeln Erwartungswert}{ewert}
Für den Erwartungswert von Zufallsvariablen $X$, $Y$ und $a,b\in\R$ gelten
folgende Rechenregeln:
\begin{align*}
\E(aX+b) &= a\E(X) + b \\
\E(X+Y) &= \E(X) + \E(Y)
\end{align*}
\end{theorem}

\begin{theorem}{Markov inequality}{markov-inequality}
Sei $X$ eine stetige \link{def:zvar}{Zufallsvariable}, $f$ eine Funktion mit
$f(X)\ge 0$ und existierndem und endlichem Erwartungswert $\E(f(X))$. Dann gilt:
\[
P\big(f(X)\ge a\big)\le \frac{\E(f(x))}{a},\quad a\in\R
\]
\end{theorem}


\subsection{Varianz}

Die Varianz ist ein Maß der Streuung einer Zufallsvariable, also wie sehr die
Werte verteilt sind.

\begin{definition}{Varianz}{varianz}
Sei $X$ eine \link{def:zvar}{Zufallsvariable} mit Zustandsraum $S=\{x_0, x_1,
...\}$ und Einzelwahrscheinlichkeiten $p_k=P(X=x_k)$. Dann heißt
\[
\Var(X) := \sum_k p_k\cdot(x_k - \E(X))^2
\]
Variaanz von $X$. Oft wird $\sigma^2$ als Zeichen für die Varianz verwendet.
\end{definition}

\begin{theorem}{Rechenregeln Varianz}{varianz}
Seien $X$, $Y$ Zufallsvariablen und $a,b\in\R$. Dann gelten folgende Rechenregeln:
\begin{align*}
\Var(aX+b) &= a^2 \Var(X) \\
\Var(X) &= \E(X^2) - \E(X)^2
\end{align*}
\end{theorem}

Im Gegensatz zum Erwartungswert gilt für die Varianz im Allgemeinen
$\Var(X+Y) \ne \Var(X) + \Var(Y)$.

\begin{theorem}{Ungleichung von Tschebyscheff}{tschebyscheff}
Sei $X$ eine Zufallsvariable mit $\E(X) = \mu$ und $\Var(X) = \sigma^2$. Dann
gilt:
\[
\forall c>0: P(|X-\mu|\ge c) \le\frac{\sigma^2}{c^2}
\]
\end{theorem}


\subsection{Kovarianz}

Die Kovarianz ist ein Maß der gemeinsamen Streuung zweier Zufallsvariablen.

\begin{definition}{Kovarianz}{kovarianz}
Sei $\zvec{X} = (X,Y)^\T$ eine Zufallsvektor. Dann heißt
\[
\Cov(X,Y) = \E(X\cdot Y) - \E(X)\cdot\E(Y)
\]
\defw{Kovarianz} von $X$ und $Y$.
\end{definition}

\begin{definition}{Kovarianzmatrix}{kovarianz-matr}
Seien $X_1, X_2, \ldots, X_n$ Zufallsvariablen. Die Matrix
\[
\Cov(\zvec{X}) = \begin{pmatrix}
  \Var(X_1) & \Cov(X_1,X_2) & \ldots & \Cov(X_1, X_n) \\
  \Cov(X_2,X_1) & \Var(X_2)  & \ldots  & \Cov(X_2, X_n) \\
    \vdots  &     \vdots   &  \ddots   &    \vdots      \\
  \Cov(X_n,X_1) & \Cov(X_n,X_2) & \ldots & \Var(X_n)
\end{pmatrix}
\]
heißt \defw{Kovarianzmatrix} von $\zvec{X}$.
\end{definition}

\begin{theorem}{Rechenregeln Kovarianz}{kovarianz}
Seien $X$ und $Y$ Komponenten des Zufallsvektors $\zvec{X}=(X,Y)^\T$ und
$a,b\in\R$. Dann gelten folgende Rechenregeln:
\begin{align*}
\Cov(X, Y) &= \Cov(Y,X) \\
\Cov(X + a, Y + b) &= \Cov(X,Y) \\
\Cov(a\cdot X, b\cdot Y) &= ab\cdot\Cov(X,Y) \\
\Cov(X, X) &= \Var(X) \\
|\Cov(X,Y)| &\le \sqrt{\Var(X)\cdot\Var(Y)} \tag{Schwarz'sche Ungleichung}
\end{align*}
\end{theorem}


\subsection{Korrelation}

Varianz und Kovarianz können unter anderem als Maß für die Korrelation zwischen
zwei Zufallsvariablen verwendet werden (das heißt wie sehr sich die
Zufallsvariablen "`gleich"' verhalten):
\begin{definition}{Korrelationskoeffizient}{korr}
Sei $\zvec{X}=(X,Y)^\T$ ein Zufallsvektor. Dann heißt
\[
f_{X,Y} = \frac{\Cov(X,Y)}{\sqrt{\Var(X)\cdot\Var(Y)}}
\]
\defw{Korrelationskoeffizient} von $X$ und $Y$.
\end{definition}

\begin{theorem}{Rechenregeln Korrelation}
Seien $X$ und $Y$ Zufallsvariablen mit Korrelationskoeffizient $f_{X,Y}$.
Dann gelten folgende Rechenregeln:
\begin{align*}
|f_{X,Y}| &\le 1 \\
X,Y \text{unabhängig} &\implies f_{X,Y} = 0 \\
|f_{X,Y}| = 1 &\implies \exists a,b\in\R, b\ne0: Y=a\cdot X + b \tag{Perfekter
linearer Zusammenhang}
\end{align*}
\end{theorem}

\subsection{Standardisierung}

\begin{definition}{Standardisierte Zufallsvariable}{std}
Sei $X$ eine Zufallsvariable mit $\E(X) = 0$ und $Var(X) = 1$. Dann heißt $Z$
\defw{standardisiert}.
\end{definition}

Eine Zufallsvariable $X$ mit $\E(X) = \mu$ und $Var(X)=\sigma^2$ kann in eine
standardisierte Zufallsvariable $\hat{X}$ überführt werden:
\[
\hat{X} = \frac{X-\mu}{\sigma}
\]


  \chapter{Wahrscheinlichkeitsverteilungen}
    \section{Diskrete Verteilungen}

\begin{definition}{Verteilungsfunktion}{vertf-disk}
Sei $X$ eine diskrete \link{def:zvar}{Zufallsvariable} mit Zustandsraum $x_0,
x_1, \ldots$. Die Funktion
\[
F_X:\R\to\R,\ F_X(z) = P(X \le z)
\]
heißt \defw{Verteilungsfunktion} der Zufallsvariable $X$. Die
Verteilungsfunktion eine Treppenfunktion, die an den Stellen $x_k$ um
$p_k = P(X=x_k)$ springt. Darum gilt:
\[
F_X(z) = \sum_{x_k\le z}P(X=x_k) = \sum_{x_k\le z} p_k
\]
\end{definition}

Eine Art Umkehrfunktion der Verteilungsfunktion ist die Quantilfunktion:
\begin{definition}{Quantilfunktion}{quantilf}
Sei $F_X(z) = P(X\le z)$ die Verteilungsfunktion einer diskreten Zufallsvariable
$X$. Dann heißt
\[
F_X^{-1}(z) = \mathrm{min}\{x\in\R: F(x) \ge z\},\quad z\in(0,1)
\]
die \defw{Quantilfunktion} von $X$.
\end{definition}

\medskip
Im Folgenden Abschnitt sei $X$ eine diskrete \link{def:zvar}{Zufallsvariable}
mit Zustandsraum $S$, $A \in S$ und $P$ eine
\link{def:verteilung}{Wahrscheinlichkeitsverteilung} dieser Zufallsvariable.


\subsection{Gleichverteilung}

Die Gleichverteilung ist eine sehr einfache Verteilung, bei der jeder Wert der
Zufallsvariable mit der gleichen Wahrscheinlichkeit auftritt:
\[
P(A) = \frac{|A|}{|S|}
\]

\subsection{Bernoulli-Verteilung ($B(p)$)}

Die Bernoulli-Verteilung beschreibt eine Zufallsvariable, die nur zwei mögliche
Zustände besitzt, hier bezeichnet als $S = \{0,1\}$. Der beliebig, aber fest
gewählte Ausgang $A$ besitzt die Wahrscheinlichkeit $0 \le p \le 1$, sodass
gilt:
\begin{align*}
P(X=1)&=p  \\
P(X=0)&=1-p
\end{align*}

\subsection{Binomialverteilung ($B(n,p)$)}

Die Binomialverteilung beschreibt die $n$-fache Durchführung eines Experiments
mit nur zwei komplementären Ausgängen werden. Die Zufallsvariable $X$ gibt an,
wie oft bei $n$-facher Wiederholung der beliebig, aber fest gewählte Ausgang $A$
eintritt. Damit kann $X$ die Werte $0, ..., n$ annehmen. Die Wahrscheinlichkeit
$p$ des Eintretens des gewählten Ausgangs bleibt dabei über alle $n$
Wiederholunge gleich. Es gilt:
\[
P(X=k) = \binom{n}{k}\cdot p^k\cdot(1-p)^{n-k}
\]

\subsection{Geometrische Verteilung ($Geo(p)$)}

Eine geometrische Verteilung entsteht durch die Wiederholung eines
Wahrscheinlichkeitsexperiments mit zwei komplementären Ausgängen. Die
Zufallsvariable $X$ beschreibt die Anzahl an Versuchen, die durchgeführt werden
müssen, bis der beliebig, aber fest gewählte Ausgang $B$ eintritt.

Der Zustandsraum von $X$ ist damit $\N_0$. Sei $p$ die Wahrscheinlichkeit, dass
Ausgang $B$ eintritt. Die Wahrscheinlichkeit, dass nach $k$ Wiederholungen der
Ausgang $B$ das erste mal auftritt, ist:
\[
P(X=k) = (1-p)^k\cdot p
\]

\subsection{Poisson-Verteilung ($Poi(\lambda)$)}

Die Poisson-Verteilung entsteht bei Vorgängen, die im Durchschnitt mit
konstanter Rate $\lambda \in (0, \infty)$ in einem beliebigen, aber
festen Zeitintervall auftreten. Die Zufallsvariable $X$ beschreibt, wie viele
Vorgänge tatächlich in dem Zeitintervall aufgetreten sind. Es gilt:
\begin{align*}
P(X=k) &= \frac{\lambda^k}{k!}\cdot\e^{-\lambda} \\
\E(X) &= \lambda
\end{align*}

\section{Stetige Verteilungen}

\begin{definition}{Wahrscheinlichkeitsdichte}{dichte}
Sei $X$ eine stetige \link{def:zvar}{Zufallsvariable} mit Zustandsraum $S$.
Eine Funktion $\rho: \R \rightarrow \R$ heißt \defw{Wahrscheinlichkeitsdichte},
wenn gilt:
\begin{align*}
  \forall x: \rho(x) \ge 0 \\
  \int \rho(x) \,\mathrm{d}x = 1
\end{align*}
\end{definition}

Die Wahrscheinlichkeitsdichte kann verwendet werden, um die Wahrscheinlichkeit,
dass X bestimmte Werte annimmt, zu berechnen ($a,b\in\R$):
\begin{align*}
  P(X < b) &= \int_{-\infty}^{b}\rho(x)\,\mathrm{d}x\\
  P(a<X<b) &= \int_{a}^{b}\rho(x)\,\mathrm{d}x\\
  P(X < b) &= \int^{\infty}_{b}\rho(x)\,\mathrm{d}x
\end{align*}

\begin{definition}{Verteilungsfunktion}{vertf}
Sei $X$ eine \link{def:zvar}{Zufallsvariable}. Die Funktion
\[F_X:\R\to\R,\ F_X(z) = P(X \le z)\]
heißt \defw{Verteilungsfunktion} von $X$.
\end{definition}

Die Verteilungsfunktion ist monoton wachsend und rechtsstetig. Weiterhin gilt:
\begin{align*}
0\le F_X(z&)\le 1 \\
F(-\infty&) = 0 \\
F(\infty&) = 1
\end{align*}

Die Verteilungsfunktion $F_X$ kann wie folgt verwendet werden, um
Wahrscheinlichkeiten bezüglich der Zufallsvariable $X$ zu berechnen:
\begin{align*}
P(X>a) = &1 - F_X(a) \\
P(X\le b) = &F_X(b) \\
P(a < X \le b) = &F_X(b) - F_X(a)
\end{align*}

Im Folgenden Abschnitt sei $X$ eine stetige \link{def:zvar}{Zufallsvariable}
mit Zustandsraum $S$, $A \in S$ und $P$ eine
\link{def:verteilung}{Wahrscheinlichkeitsverteilung} dieser Zufallsvariable.


\subsection{Gleichverteilung ($U(a,b)$)}
\label{vert-gleich}

Die gleichverteilte Zufallsvariable $X$ nimmt die Werte $S=(a,b)$ an. Für die
Wahrscheinlichkeitsdichte gilt:
\[
\rho(x) = \frac{1}{b-a}\cdot\mathbb{I}_{(a,b)}(x)
\]
Dabei bezeichnet $\mathbb{I}_{(a,b)}$ die \defw{Indikatorfunktion}, die Werte im
Intervall von $(a,b)$ auf $1$ und alle anderen Werte auf $0$ abbildet.

\subsection{Exponentialverteilung ($Exp(\lambda)$)}
\label{vert-exp}

Für eine Exponentialverteilung mit konstanter Ereignisrate $\lambda>0$ gilt:
\begin{align*}
\rho(x) &= \lambda\cdot \e^{-\lambda x}\cdot\mathbb{I}_{(0, \infty)} \\
F(z) &= 1 - \e^{-\lambda\cdot z}
\end{align*}

\subsection{Normalverteilung ($N(\mu, \sigma^2)$)}

In der Natur kommen Normalverteilungen vor wenn sich eine große Anzahl
unabhängiger Verteilungen überlagern. Für die Wahrscheinlichkeitsdichte gilt:
\[
\rho(x) = \frac{1}{\sqrt{2\pi\sigma^2}}\cdot exp(-\frac{(x-\mu)^2}{2\sigma^2})
\]

\subsection{Standardnormalverteilung ($N(0,1)$)}
\label{vert-stdnormal}

Eine standardnormalverteilte Zufallsvariable nimmt im Mittel den Wert $0$ mit
einer Varianz von $1$ an. Es gilt:
\[
\rho(x) = \frac{1}{\sqrt{2\pi}}\cdot exp(-\frac{x^2}{2})
\]

Die \link{def:vertf}{Verteilungsfunktion} der Standardnormalverteilung wird mit $\Phi(z)$ bezeichnet.

Ist $X$ normalverteilt mit $\mu$ und $\sigma^2$, dann ist die Zufallsvariable
\[
Z=\frac{X-\mu}{\sigma}
\]

standardnormalverteilt.


  \chapter{Generierung von Zufallszahlen}
    Dieses Kapitel behandelt Methoden zur Generierung von diskreten und stetigen
mit beliebiger Verteilung Zufallszahlen. Wir nehmen an, mehrere
Pseudo-Zufallszahlengenerator (PRNG) $U_1, U_2, \ldots$ zur Verfügung zu haben,
die unabhängig voneinander $U(0,1)$ verteilte Zufallszahlen erzeugen.

PRNG sind nicht "`echt"' zufällig, da eine Folge von Zufallszahlen durch den
Startwert des Generators bestimmt ist und damit auch reproduziert werden kann.
Zusätzlich ist ein PRNG periodisch, sodass sich die generierten Zahlen ab einem
bestimmten Punkt wiederholen. Trotz dieser Eigenschaften sind (gute) PRNG sehr
gut für praktische Anwendungen geeignet, da sie sehr schnell Zufallszahlen
erzeugen können und die erzeugten Zahlen bei einem zufälligen Startwert nicht
von tatsächlich zufälligen Zahlen (z.B. bestimmt durch einen physikalischen
Prozess) unterscheidbar sind.

\section{Zufallszahlen mit diskreter Verteilung}

Ziel ist es, zufällig Zahlen zu erzeugen, die wie die Zufallsvariable $X$ mit
Zustandsraum $S = \{x_0, x_1, \ldots\}$ und zugehörigen Wahrscheinlichkeiten
$p_k = P(X=x_k)$ verteilt sind. Wir bestimmen im Folgenden ein Verfahren, mit
dem wir die Zufallszahlen $U_1\sim U(0,1)$ in die Zielverteilung umformen
können.

Das einfachste Vorgehen ist die proportionale Aufteilung des Intervalls $[0,1]$
gemäß der Wahrscheinlichkeiten $p_k$. Jedem $p_k$ wird der Bereich von $s_{k-1}$
bis $s_k$ zugewiesen, der genau $p_k$ breit ist:
\[
s_{k} = \sum_{i=0}^k p_k
\]
Fällt die von $U_1$ generierte Zufallszahl in das Intervall $(s_{k-1}, s_k)$, so
wird $x_k$ ausgegeben. Damit werden Zufallszahlen in der Zielverteilung erzeugt.

Folgender Algorithmus implementiert diese Methode und erzeugt $N$ wie $X$
verteilte Zufallszahlen:

\begin{algorithm}[h!]

\For{$1$ \KwTo $N$}{
  Setze $k=0$, $s=p_k$\;
  Erzeuge $u\sim U(0,1)$\;
  \While{$u > s$}{
    Setze $k \leftarrow k+1$\;
    Setze $s \leftarrow s+p_k$\;
  }
  Gib $x_k$ aus\;
}

\caption{Erzeugung diskreter Zufallszahlen}\label{algo:zz-diskret}
\end{algorithm}

Bei diesem Algorithmus sind in der inneren Schleife $\E(X)$ Durchläufe
notwendig (zumindest sofern $x_0 = 0$, $x_1=1$, usw. gilt)
um das $x_k$, in dessen Intervall $u$ fällt, zu finden.
Besitzt $X$ einen großen Zustandsraum, kann das die Geschwindigkeit des
Algorithmus erheblich verschlechtern.

Man kann die Laufzeiteigenschaften jedoch
verbessern, indem man $X$ auf eine Zufallsvariable $\hat{X}$ abbildet, deren
Werte $\hat{x}_k$ nach $\hat{p}_k$ absteigend geordnet sind. Somit steigt die
Wahrscheinlichkeit, bereits nach wenigen Durchläufen der inneren Schleife das
richtige $\hat{x}_k$ gefunden zu haben.

Die Idee hinter dem Algorithmus kann weiter verallgemeinert werden. Dafür
betrachten wir die \link{def:quantilf}{Quantilfunktion} von $X$, die die
\link{def:vertf-disk}{Verteilungsfunktion} umkehrt. Die Quantilfunktion gibt
uns also für Werte zwischen $0$ und $1$ zurück, was das $x_k$ zwischen $s_{k-1}$
und $s_k$ ist. Wir nutzen also (implizit) in Algorithmus \ref{algo:zz-diskret}
bereits die Quantilfunktion.

\begin{theorem}{Inversionsprinzip für diskrete Variablen}{invp}
Sei $X$ eine diskrete Zufallsvariable mit \link{def:vertf-disk}{Verteilungsfunktion}
$F_X$ und \link{def:quantilf}{Quantilfunktion}
$F_X^{-1}$. Weiterhin sei $U$ eine $U(0,1)$-verteilte Zufallsvariable. Dann
besitzt $F_X^{-1}(U)$ die gleiche Verteilung wie $X$, das heißt:
\[
P\big(F_X^{-1}(U)\le z\big) = F_X(z)
\]
\end{theorem}
Das ist die Formalisierung von unserem Ansatz, die Wahrscheinlichkeiten $p_k$
der Zufallsvariable $X$ proportional im Intervall $(0,1)$ zu betrachten.

\section{Inversionsmethode}\label{algo:inv-methode}

\begin{theorem}{Transformationssatz}{trafo}
Sei $X$ eine stetige Zufallsvariable mit \link{def:dichte}{Dichte} $\rho_X$ und
Zustandsraum $S$. Sind $a,b\in\R$ und $a \ne 0$, so besitzt die Zufallsvariable
$Y = a\cdot X + b$ die Dichte
\[
\rho_Y(z) = \rho_X\Big(\frac{z-b}{a}\Big)
\]
Ist die Funktion $g:S\to\R$ streng monoton, so hat die Zufallsvariable $Y=g(X)$
die Dichte
\[
\rho_Y(z) = \rho_X\big(g^{-1}(z)\big)\cdot\big|\big(g^{-1}\big)^\prime(z)\big|
\]
\end{theorem}

Die Wahrscheinlichkeitsdichte einer verteilte Zufallsvariable $Y$ kann auf die
Wahrscheinlichkeitsdichte einer anderen Zufallsvariable $X$ zurückgeführt
werden. Genau das wenden wir jetzt mit der Inversionsmethode an: Die
Zufallsvariable $Y$ soll eine beliebige Verteilung besitzten, die wir aus
gleichverteilten Zufallszahlen erzeugen. Dafür benötigen wir nur noch eine
geeignete Funktion $g$, sodass $Y=g(X)$ gilt. Das erreichen wir durch das
Inverse der Verteilungsfunktion:

\begin{theorem}{Inversionsmethode}{inversionsm}
Sei $U\sim U(0,1)$ eine gleichverteilte Zufallsvariable und $Y$ eine
beliebige stetige Zufallsvariable mit \link{def:vertf}{Verteilungsfunktion}
$F_Y$. Dann gilt:
\[
F_Y^{-1}(U) \sim Y
\]
\end{theorem}

\begin{example}{Exponentialverteilung}{exp}
Wir wollen Zufallszahlen $X\sim Exp(\alpha)$ erzeugen. $X$ besitzt die Dichte
\[
\rho(x) = \begin{cases}
\alpha \e^{-\alpha x} & x > 0 \\
0 & \text{sonst}
\end{cases}
\]
mit der Verteilungsfunktion
\[
F_X(z) = \begin{cases}
1- \e^{-\alpha z} & z > 0 \\
0 & \text{sonst}
\end{cases}
\]
Bei Einschränkung der Verteilungsfunktion auf den Zustandsraum $(0, \infty)$ von
$X$ ergibt sich:
\[
F_X: (0,\infty)\to(0,1):x\mapsto1-\e^{-\alpha x}
\]
Durch Umstellen ergibt sich folgende Vorschrift für die Umkehrfunktion:
\[
F_X^{-1}(y) = -\frac{1}{\alpha} \mathrm{ln}(1-y)
\]
Damit können wir mit der Inversionsmethode aus einer gleichverteilten
Zufallsvariable $U$ Zufallszahlen erzeugen, die wie $X$ verteilt sind:
\[
F_X^{-1}(U) = -\frac{1}{\alpha} \mathrm{ln}(1-U) \sim X
\]
\end{example}

Die Inversionsmethode ist ein effizientes Verfahren, um beliebig verteilte
Zufallszahlen zu erzeugen. Problematisch ist jedoch, dass wir die
Verteilungsfunktion invertieren müssen, was nicht immer möglich ist.

\section{Annahme-Verwerfungs-Methode}

\begin{wrapfigure}{r}{0.5\textwidth}
\centering
\includegraphics[width=0.4\textwidth]{av-methode}
\end{wrapfigure}
Die Annahme-Verwerfungs-Methode ist ein geometrischer Ansatz, der zufällig
Punkte in einem rechteckigen Bereich generiert, der die
\link{def:dichte}{Dichtefunktion} der gewünschten Verteilung einrahmt.
Notwendige Voraussetzung dafür ist, dass ein endlich großes, die Dichtefunktion
umgebendes Rechteck gefunden werden kann.

Die zufälligen Punkte werden durch skalierte, gleichverteilte Zufallszahlen
bestimmt. Liegt der Punkt unterhalb des Graphs der Dichtefunktion, wird der
X-Wert des Punkts ausgegeben. Damit entspricht die Wahrscheinlichkeit, dass ein
bestimmter Wert als Zufallszahl ausgegeben wird, genau dem Wert der
Wahrscheinlichkeitsdichte an diesem Punkt.

Der folgende Algorithmus gibt eine Folge von $N$ Zufallszahlen mit der gewünschten
Verteilung aus:

\begin{algorithm}[h!]

\For{$1$ \KwTo $N$}{
  Erzeuge $x\sim U(a,b)$\;
  Erzeuge $y\sim U(0,1)$\;
  \If{$y\cdot c \le \rho(x)$}{
    gib $x$ aus\;
  }
}

\caption{Annahme-Verwerfungs-Methode}\label{algo:av-methode}
\end{algorithm}

\subsection{Importance-Sampling}
\label{algo:imp-samp}

Die Annahme-Verwerfungs-Methode funktioniert dann besonders gut, wenn die Fläche
unter der Dichtefunktion im Vergleich zum einhüllenden Rechteck möglichst gering
ist. In diesem Fall liegen nur wenige Punkte oberhalb der Dichte und müssen
verworfen werden. Ist das einhüllende Rechteck im Vergleich jedoch sehr groß,
beispielsweise weil die Wahrscheinlichkeitsdichte sehr ungleich verteilt ist,
werden viele Punkte verworfen, sodass mehr Zeit für die Erzeugung einer festen
Anzahl an Zufallszahlen erforderlich ist.

Beim Importance-Sampling wird statt eines einhüllenden Rechtecks eine einhüllende
Funktion $h(x)$ verwendet, die weniger Platz als das Rechteck "`verschwendet"'.
Dafür ersetzen wir in Algorithmus \ref{algo:av-methode} die obere Schranke $c$
durch den Wert der Einhüllenden $h(x)$.

Da die Einhüllende $h(x)$ jedoch im Allgmeinen nicht konstant ist, müssen wir
diese Verzerrung korrigieren. Dafür generieren wir die x-Werte der zufälligen
Punkte in der Verteilung der Einhüllenden. Die Einhüllende $h(x)$ besitzt
folgende \link{def:dichte}{Dichte}:
\[H(z) = \frac{1}{\gamma}\cdot\int_a^z h(x) \mathrm{d}x\]
Die Konstante $\gamma = \int_a^b h(x)\mathrm{d}x$ stellt die Normiertheit sicher.

Durch die Inverse $H^{-1}$ der Verteilungsfunktion können mit der
\link{satz:inversionsm}{Inversionsmethode} der Einhüllenden entsprechend
verteilte Zufallszahlen erzeugt werden (da $h(x)$ frei gewählt werden kann, ist
das Invertieren in der Regel kein Problem).

Der entsprechende Algorithmus sieht dann so aus:

\begin{algorithm}[h!]

\For{$1$ \KwTo $N$}{
  Erzeuge $u\sim U(a,b)$\;
  Setze $x = H^{-1}(u)$\;
  Erzeuge $y\sim U(0,1)$\;
  \If{$y\cdot h(x) \le \rho(x)$}{
    gib $x$ aus\;
  }
}

\caption{Annahme-Verwerfungs-Methode mit Importance Sampling}\label{algo:av-methode-is}
\end{algorithm}

\section{Normalverteilte Zufallszahlen}

\begin{theorem}{Transformationssatz (2D)}{trafo-2d}
Seien $\zvec{X} = (X, Y)^\T$ und $\zvec{Z} = (S,T)^\T$
\link{def:zvektor}{Zufallsvektoren} mit gemeinsamer Dichte
$\rho_{X,Y}$ bzw. $\rho_{S,T}$. Weiterhin sei $g$ eine bijektive Funktion mit
Umkehrfunktion $g^{-1}$, die beide Zufallsvektoren aufeinander
abbildet:
\begin{align*}
g: \R^2\to\R^2: \begin{pmatrix}x\\y\end{pmatrix}\mapsto
  \begin{pmatrix}g_1(x,y)\\g_2(x,y)\end{pmatrix} =
  \begin{pmatrix}s\\t\end{pmatrix} \\
g^{-1}: \R^2\to\R^2: \begin{pmatrix}s\\t\end{pmatrix}\mapsto
  \begin{pmatrix}h_1(s,t)\\h_2(s, t)\end{pmatrix} =
  \begin{pmatrix}x\\y\end{pmatrix}
\end{align*}
Dann gilt für die Wahrscheinlichkeitsdichte von $\zvec{Z}$:
\[
\rho_{S,T}(s,t) = \rho_{X,Y}\big(g^{-1}(s,t)\big)\cdot\Det(J) =
  \rho_{X,Y}(h_1(s,t), h_2(s,t)) \cdot \Det(J)
\]
wobei $J$ die Jacobi-Matrix\more{wiki-jacobi} von $g^{-1}$ ist.
\end{theorem}

Ähnlich wie der \link{satz:trafo}{"`normale"' Transformationssatz} formalisiert der
2D-Tranformationssatz, wie sich die Wahrscheinlichkeitsdichte bei der Abbildung
einer Zufallsvariablen durch eine Funktion $g$ verändert. Das ist ein hilfreiches
Werkzeug bei der Erzeugung von Zufallszahlen, da wir im Allgemeinen
$U(0,1)$-verteilte Zufallszahlen in komplexere Verteilungen umformen möchten.

\subsection{Box-Muller-Methode}

Die Box-Muller-Methode nutzt den Transformationssatz und die Polardarstellung
von Koordinaten zur Erzeugung zweidimensional normalverteilter Zufallszahlen.

Seien $X\sim N(0,1)$ und $Y\sim N(0,1)$ \link{vert-stdnormal}{standardnormalverteilte}
und \link{def:zvar-unabh}{unabhängige} Zufallsvariablen. Damit bestehen folgende
Dichtefunktionen:
\begin{align*}
\rho_X(x) = \frac{1}{\sqrt{2\pi}}\cdot\exp\Big(-\frac{x^2}{2}\Big)\\
\rho_Y(y) = \frac{1}{\sqrt{2\pi}}\cdot\exp\Big(-\frac{y^2}{2}\Big)
\end{align*}
Da die Zufallsvariablen unabhängig sind, können wir Satz \ref{satz:randd}
anwenden um die gemeinsame Dichte zu bestimmen:
\[
\rho_{X,Y}(x,y) = \rho_X(x)\cdot\rho_Y(y) =
\frac{1}{2\pi}\cdot\exp\Big(-\frac{x^2+y^2}{2}\Big)
\]
Wir wenden den Transformationssatz mit einer Funktion $g$ an, die unsere
kartesischen Koordinaten $X,Y$ in Polarkoordinaten \more{wolfram-polar} $Z,\Phi$
umwandelt:
\begin{align*}
g: \R^2 \to \R^2: (x,y) &\mapsto (r^2 =x^2+y^2,\phi=\atan\frac{y}{x}) \\
g^{-1}: \R^2 \to \R^2: (z,\phi) &\mapsto (x=\sqrt{z}\cdot\cos\phi,y=\sqrt{z}\cdot\sin\phi)
\end{align*}
Dabei wird $z=r^2$ definiert. Als Jacobi-Matrix von $g^{-1}$ ergibt sich
\[
\renewcommand*{\arraystretch}{1.8}
J = \begin{pmatrix}\pd{h_1}{z} & \pd{h_1}{\phi} \\ \pd{h_2}{z} & \pd{h_2}{\phi}\end{pmatrix}
 = \begin{pmatrix}\frac{1}{2\sqrt{z}}\cos\phi & -\sqrt{z}\sin\phi \\
                  \frac{1}{2\sqrt{z}}\sin\phi &  \sqrt{z}\cos\phi\end{pmatrix}
\]
mit Determinante (Anwendung des trigonometrischen Pythagoras)
\[
\det J = \frac{1}{2}\cos^2\phi + \frac{1}{2}\sin^2\phi = \frac{1}{2}
\]
Durch Anwendung des \link{satz:trafo-2d}{Transformationssatzes} ergibt sich die
Verteilung $\rho_{Z,\Phi}$ der durch $g$ in Polarkoordinaten umgewandelten
Werte:
\begin{align*}
\rho_{Z,\Phi} &= \rho_{X,Y}(\sqrt{z}\cos\phi,\sqrt{z}\sin\phi)\cdot\det J \\
  &= \frac{1}{2\pi}\cdot
        \exp\Big(-\frac{(\sqrt{z}\cos\phi)^2+(\sqrt{z}\sin\phi)^2}{2}\Big)
        \cdot\det J \\
  &= \frac{1}{2\pi}\cdot\frac{1}{2}\exp\Big(-\frac{z}{2}\Big)
\end{align*}
Diese Verteilung besteht aus zwei bekannten Wahrscheinlichkeitsdichten: Der
erste Teil ist eine \link{vert-gleich}{Gleichverteilung} auf $(0,2\pi)$,
der zweite Teil ist eine \link{vert-exp}{Exponentialverteilung} mit
Parameter $\alpha=0.5$ (vgl. Beispiel \ref{bsp:exp}). Damit könnten wir
zweidimensional normalverteilte Zufallsvariablen erzeugen, wenn wir unabhängige
Zufallsvariablen $\Phi\sim U(0,2\pi)$ und $Z\sim Exp(0.5)$ erzeugen können.
Da wir annehmen, nur Zufallszahlen $U_1, U_2, ... \sim U(0,1)$ zur Verfügung zu
haben, wenden wir die \link{satz:inversionsm}{Inversionsmethode} an:
\begin{align*}
\Phi &= 2\pi \cdot U_1 \\
Z &= -2\ \ln(1-U_2)
\end{align*}
Durch Auflösung der Polarkoordinaten und $z = r^2$ erhält man:
\begin{align*}
X = \sqrt{-2\,\ln(1-U_2)}\cdot\cos(2\pi\cdot U_1) \\
Y = \sqrt{-2\,\ln(1-U_2)}\cdot\sin(2\pi\cdot U_1)
\end{align*}

Damit sind die Zufallsvariablen $X,Y \sim N(0,1)$ standardnormalverteilt. Um
$N(\mu, \sigma^2)$-verteilte Zufallsvariablen zu erzeugen, kann eine lineare
Transformation angewandt werden.

\medskip
Damit ergibt sich folgender Algorithmus zur Erzeugung von unabhängigen,
normalverteilten Zufallszahlen $X~N(\mu, \sigma^2)$. Der Algorithmus erzeugt
$N$ Paare von normalverteilten Zufallszahlen:

\begin{algorithm}[h!]

\For{$1$ \KwTo $N$}{
  Erzeuge $u\sim U(0,1)$\;
  Erzeuge $v\sim U(0,1)$\;
  Setze $z_1 = \sqrt{-2\,\ln(1-u)}\cdot\cos(2\pi\cdot v)$\;
  Setze $z_2 = \sqrt{-2\,\ln(1-u)}\cdot\sin(2\pi\cdot v)$\;
  Setze $x_1 = \sigma\cdot z_1 + \mu$\;
  Setze $x_2 = \sigma\cdot z_2 + \mu$\;
  Gib $x_1$ aus\;
  Gib $x_2$ aus\;
}

\caption{Box-Muller-Methode}\label{algo:box-muller}
\end{algorithm}

\subsection{Methode von Box, Muller, Marsaglia (Polarmethode)}

Die Polarmethode ist eine weitere Methode zur Erzeugung normalverteilter
Zufallszahlen, die die Auswertung von trigonometrischen Funktionen und des
Logarithmus vermeidet und daher schneller als die Box-Muller-Methode ist.

Der folgende Algorithmus beschreibt die Polarmethode:
\begin{algorithm}[h!]
\SetKw{Continue}{continue}

\For{$1$ \KwTo $N$}{
  Erzeuge $u\sim U(0,1)$\;
  Erzeuge $v\sim U(0,1)$\;
  Setze $x=2u-1$\;
  Setze $y=2v-1$\;
  Setze $s = x^2 + y^2$\;
  \If{$s\ge 1$}{
    \Continue
  }
  \eIf{$s \ne 0$}{
    Setze $z_1 = x\sqrt{\frac{-2\,\ln s}{s}}$\;
    Setze $z_2 = y\sqrt{\frac{-2\,\ln s}{s}}$\;
  }{
    Setze $z_1 = z_2 = 0$\;
  }
}

\caption{Box-Muller-Marsaglia-Methode (Polarmethode)}
\label{algo:box-muller-marsaglia}
\end{algorithm}

\section{Mehrdimensionale Zufallszahlen}

Bei der Erzeugung von Zufallsvektoren müssen die bedingte Abhängigkeiten
zwischen den jeweiligen Zufallsvariablen berücksichtigt werden.

Zufallsvektoren können in mehreren Schritten komponentenweise erzeugt werden.

Sei $\zvec{X} = (X,Y)^\T$ ein \link{def:zvektor}{Zufallsvektor} mit
\link{def:vert-zvektor}{Einzelwahrscheinlichkeiten} $p_{ij}$. Zuerst erzeugen
wir die erste Komponente des Vektors gemäß der \link{def:randv}{Randverteilung}
von $X$. Entsprechend dieser Verteilung wird mit einer geeigneten Methode
(Inversionsmethode, Annahme-Verwerfungs-Methode o.Ä.) ein $x$ ermittelt.

Durch die Randverteilung von $X$ können wir die \link{def:bedd}{bedingte
Einzelwahrscheinlichkeit} von $Y$, also $P(Y=y_i|x)$ bestimmt werden und ein
$y$ gemäß der Verteilung zufällig ermittelt werden.

Auf diese Weise können n-dimensionale Zufallszahlen erzeugt werden.

\subsection{Mehrdimensional normalverteilte Zufallszahlen}

\begin{figure}[h!]
\centering
\includegraphics[width=10cm]{normal-2d}
\caption{Visualisierung einer zweidimensionalen Normalverteilung}
\end{figure}

\begin{definition}{Mehrdimensionale Normalverteilung}{ndim-normal}
Sei $\zvec{X} = (X_1, \ldots,X_d)^\T$ ein Zufallsvektor,
$\vec{\mu} = (\mu_1, \ldots, \mu_d)^\T \in\R^d$ ein Vektor der
Erwartungswerte und $\Sigma = (\sigma_{ij})_{i,j=1,\ldots,d}$
\link{def:kovarianz-matr}{Kovarianzmatrix} von $\zvec{X}$. Besitzt der
Zufallsvektor die Dichte
\[
\rho_{\zvec{X}} = \frac{1}{\sqrt{(2\pi)^d\cdot\det\Sigma}}\cdot
  \exp\Big(-\frac{1}{2}(\vec{x} - \vec{\mu})^\T\ \Sigma^{-1}(\vec{x}-\vec{\mu})\Big)
\]
so heißt er $\zvec{X}$ \defw{normalverteilt} (Schreibweise: $\zvec{X}\sim
N_d(\vec\mu,\Sigma)$). Jede Komponente des Vektors ist normalverteilt, das heißt
es gilt $\forall 0\le i\le d: X_i \sim N(\mu_i,\sigma_{ii})$.
\end{definition}

\begin{theorem}{Transformation einer mehrdimensionalen
Normalverteilung}{trafo-normal-ndim}
Sei $\zvec{X}$ ein normalverteilter Zufallsvektor von $d$ Komponenten mit
Erwartungswertvektor $\vec\mu$ und Kovarianzmatrix $\Sigma$. Sei $A$ eine
$n\times d$-Matrix ($n\in\N$). Dann ist die Transformation $\zvec{Y}= A\zvec{X}$
wieder normalverteilt mit
\begin{align*}
\E(\zvec{Y}) &= A\vec\mu \\
\Cov(\zvec{Y}) &= A\Sigma A^\T
\end{align*}
\end{theorem}

Diesen Satz können wir für die Erzeugung mehrdimensionaler Normalverteilungen
verwenden. Dafür erzeugen wir einen Zufallsvektor $\zvec{X} \sim
N_d(\vec{0},\mathds{1})$,
den wir so mit einer zu bestimmenden Matrix $A$ transformieren, dass gilt:
\[
\Cov(A\zvec{X}) = A\cdot \mathds{1}\cdot A^\T = A\cdot A^\T
\]

Dadurch ergibt sich folgender Algorithmus für die Erzeugung von $N$
wie $N_d(\vec\mu, \Sigma)$ verteilten Zufallsvektoren:
\begin{algorithm}[h!]

Finde $A\in\R^{(d\times d)}$ sodass $A\cdot A^\T = \Sigma$\;

\For{$1$ \KwTo $N$}{
  Erzeuge $z_1, \ldots,z_d$ unabhängig verteilt wie $N(0,1)$\;
  Setze $(x_1, \ldots, x_d)^\T = \vec\mu + A\cdot (z_1, \ldots, z_d)^\T$\;
  Gib $(x_1, \ldots, x_d)^\T$ aus\;
}

\caption{Erzeugung von normalverteilten Zufallsvektoren}
\label{algo:zz-normal-ndim}
\end{algorithm}

Um $z_1, \ldots, z_d$ zu erzeugen kann zum Beispiel die
\link{algo:box-muller-marsaglia}{Polarmethode} verwendet werden. Die Bestimmung
der Matrix $A$ entspricht der sogenannten Matrixwurzel von $\Sigma$, die durch
die Orthonormalbasis der Eigenvektoren bestimmt werden
kann\more{wiki-matrixwurzel}.


  \chapter{Markov-Ketten}
    Markov-Ketten dienen zur Beschreibung von Prozessen, deren zukünftiger Zustand
nur durch den letzten Zustand bestimmt wird. Markov-Prozesse werden darum auch
als "`gedächtnislos"' bezeichnet.

\begin{definition}{Stochastischer Prozess}{stochp}
Sei $T=\N_0$ und $S \subset\R$. Für jedes $t\in T$ sei $X_t$ eine
\link{def:zvar}{Zufallsvariable} mit Zustandsraum $S$. Dann heißt die Familie
\[
\big(X_t\big)_{t\in T}
\]
\defw{stochastischer Prozess} mit Zustandsraum $S$ und diskreter Zeit. Ist
$T = [0, \infty)$ handelt es sich um einen \defw{stetigen stochastischen Prozess}.
\end{definition}

\begin{definition}{Markov-Kette}{mk}
Ein stochastischer Prozess $\big(X_t\big)_{t\in \N_0}$ mit Zustandsraum $S$
heißt \defw{Markov-Kette} falls für alle $n\in\N_0$ und $k,l,x_0,x_1,...,x_n \in S$
gilt:
\[
P\big(X_{n+1} = l | X_{n}=k, X_{n-1}=x_{n-1},...,X_0=x_0\big) =
P\big(X_{n+1}=l|X_n=k\big)
\]
Diese Wahrscheinlichkeit heißt Übergangswahrscheinlichkeit und wird mit $p(k,l)$
bezeichnet.
\end{definition}

Die Definition der Markov-Kette beschreibt genau die Eigenschaft der
"`Gedächtnislosigkeit"': Die Wahrscheinlichkeit, in den nächsten Zustand zu
wechseln hängt lediglich davon ab, wo man sich gerade befindet (und nicht von
den Schritten davor).

\begin{definition}{Übergangsmatrix einer Markovkette}{mk-matr}
Sei $(X)_{n\in N_0}$ eine \link{def:mk}{Markovkette} mit Zustandsraum $S$. Die
Übergangswahrscheinlichkeiten $p(i,j)$ mit $i,j\in S$ lassen sich als Matrix
anordnen:
\[
\Pi = \big(p(i,j)\big)_{i,j\in S}
\]
Diese Matrix wird als \defw{Übergangsmatrix} der Markov-Kette bezeichnet.
\end{definition}

Jede Zeile der Übergangsmatrix enthält die Wahrscheinlichkeiten, in den Zustand
der jeweiligen Spalte überzugehen. Die Matrix ist eine sogenannte
\defw{stochastische Matrix}, das heißt alle Einträge besitzen Werte zwischen $0$
und $1$ und die Summe jeder Zeile ist $1$.

\begin{definition}{Verteilung einer Markovkette}{mk-vert}
Sei $(X)_{n\in N_0}$ eine \link{def:mk}{Markovkette} mit Zustandsraum $S$. Dann
heißt der Zeilenvektor mit $m\in \N_0$
\[
\pi_m = \big(P(X_m=s_1), ...,P(X_m=s_n)\big),\ s_1, ..., s_n \in S
\]
Verteilung der Markovkette zur Zeit $m$.
\end{definition}

Die Komponente einer Verteilung $\pi$ für einen Zustand $s$ wird mit $\pi(s)$
bezeichnet.

\begin{theorem}{Berechnung einer Markovkette}{mk-ber}
Sei $S$ eine diskrete Menge, $\Pi = \big(p(k,l)\big)_{k,l\in S}$ eine
stochastische Matrix auf $S$ und $\pi_0 = \big(p_0(k)\big)$ die
\link{def:mk-vert}{Verteilung} der Zustände zu Beginn der Betrachtung. Dann ist
die Verteilung $\pi_n$ nach $n$ Schritten berechenbar durch
\[
\pi_n = \pi_0\cdot\Pi^n
\]
Die Matrix $\Pi^n$ gibt die Wahrscheinlichkeit an, in $n$ Schritten von
Zustand $i$ in Zustand $j$ überzugehen.
\end{theorem}

Damit hängt die Wahrscheinlichkeit, sich nach einer festen Anzahl an Schritten
in einem bestimmten Zustand zu befinden, neben den Übergangswahrscheinlichkeiten
$\Pi$ nur von der Anfangsverteilung ab.

\begin{theorem}{Chapman-Kolmogorow-Gleichung}{chapman}
Sei $(X)_{n\in N_0}$ eine Markovkette mit Zustandsraum $S$. Dann kann die
Wahrscheinlichkeit, in $n+m$ Schritten von Zustand $i$ in Zustand $k$ zu
wechseln gleich der Summe der Pfade über alle möglichen Zwischenstationen:
\[
P(X_{n+m}=k|X_0=i) = \sum_{j\in S}P(X_{n+m}=k|X_n=j)\cdot P(X_n=j|X_0=i)
\]
\end{theorem}

Das lässt sich durch die Definition der Matrixmultiplikation zeigen
(\href{https://de.wikipedia.org/wiki/Chapman-Kolmogorow-Gleichung}{Wikipedia}).

\section{Eigenschaften}

\begin{definition}{Pfad}{mk-pfad}
Ein konkreter Folge von Zuständen einer Markovkette $(X)_{n\in N_0}$ wird
als \defw{Pfad} bezeichnet.
\end{definition}

\subsection{Zustandsklassen}

Die Zustände von Markovketten können zum Beispiel nach ihrer Erreichbarkeit
untereinander unterschieden werden.

\begin{definition}{Interaktionsgraph}{mk-igraph}
Sei $(X)_{n\in N_0}$ eine Markovkette mit Zustandsraum $S$. Der gerichtete Graph
$G=(V,E)$ mit Kanten $V=S$ und $E = \{(x,y) \in S\times S\ |\ p(x,y) \ne 0\}$ wird
als \defw{Interaktionsgraph} bezeichnet.
\end{definition}

Der Interaktionsgraph beschreibt also die direkten
Verbindungen der Zustände im Zustandsraum.

\begin{definition}{Erreichbarkeit}{mk-erreichbar}
Sei $(X)_{n\in N_0}$ eine Markovkette mit Zustandsraum $S$. Ein Zustand $y\in S$
heißt erreichbar von $x\in S$, falls es ein $n>0$ gibt, sodass gilt:
\[
P(X_n=y|X_0=x) > 0
\]
Erreichbarkeit kann auch durch die \link{def:mk-matr}{Übergangsmatrix} $\Pi$
definiert werden:
\[
\exists n\in\N: \Pi^n(x,y) \ne 0
\]
\end{definition}

Ist $y$ von $x$ erreichbar, existiert im Interaktionsgraph ein Pfad von $x$ zu
$y$. Dieser Pfad muss nicht direkt sein, sondern kann auch über andere Zustände
führen. Wir verwenden für diese Erreichbarkeit die Schreibweise $x\to y$.

\begin{definition}{Verbundene Zustände}{mk-verb}
Sei $S$ der Zustandsraum einer Markovkette. Die Zustände $x,y\in S$ heißen
\defw{verbunden}, wenn gilt:
\[
x\to y \land y \to x
\]
Verbundene Zustände werden durch das Zeichen $\lr$ gekennzeichnet
($x \lr y$).
\end{definition}

Die Verbundenheit von Zuständen ist eine Relation, die \emph{reflexiv} (jeder
Zustand ist mit sich selbst verbunden), \emph{symmetrisch} (verbundene
Zustände sind auch in der "`Gegenrichtung"' verbunden) und \emph{transitiv}
(sind $a$ mit $b$ und $b$ mit $c$ verbunden, ist auch $a$ mit $c$
verbunden). Damit ist die Relation $\lr$ eine Äquivalenzrelation\more{mfnf-ä-rel}.
Das bedeutet insbesondere, dass die Zustandsmenge $S$ durch die Verbundenheitsrelation in
Äquivalenzklassen zerlegt wird, in denen jeder Zustand mit jedem anderen Zustand
verbunden ist.

Zwischen Äquivalenzklassen kann sich nicht beliebig bewegt werden; insbesondere
kann eine einmal verlassene Klasse nicht wieder erreicht werden. (Gäbe es einen
"`Weg zurück"', wären auch jedes Paar von Zuständen aus beiden Klassen
miteinander verbunden. Das ist ein ein Widerspruch, denn dann müssten diese
Zustände ja in \emph{einer} gemeinsamen Äquivalenzklasse liegen.)

Um die Zustände einer Markovkette zu klassifizieren, beginnt man bei einem
beliebigen Zustand und bestimmt dessen transitive Hülle bezüglich $\lr$, also
alle Zustände die mit dem gewählten Zustand verbunden (nicht nur erreichbar!)
sind. Damit ist eine Klasse der Markovkette gefunden. Dieses Vorgehen
wird dann mit nicht klassifizierten Zuständen wiederholt, bis alle Zustände
klassifiziert sind.

\begin{example}{Bestimmung der Zustandsklassen}{}
Gesucht sind die Zustandsklassen eine Markovkette mit folgendem Interaktionsgraph:

\begin{centering}
\begin{tikzpicture}
  \node[v] (1) at (2,1) {1};
  \node[v] (2) at (4,2) {2};
  \node[v] (3) at (6,2) {3};
  \node[v] (6) at (5,0) {6};
  \node[v] (4) at (8,1) {4};
  \node[v] (5) at (10,1) {5};

  \draw[e] (1) to[loop above] ();
  \draw[e] (6) -- (1);
  \draw[e] (6) -- (3);
  \draw[e] (2) to[bend right] (3);
  \draw[e] (3) to[bend right] (2);
  \draw[e] (2) to[loop above] ();
  \draw[e] (2) -- (6);
  \draw[e] (6) -- (4);
  \draw[e] (4) to[bend left] (5);
  \draw[e] (5) to[bend left] (4);
\end{tikzpicture}
\end{centering}

Zustand 1 ist mit keinen weiteren Zuständen verbunden.
Zustand 2 ist mit Zuständen 3 und 6 verbunden, aber z.B. nicht mit Zustand 4
oder 1, da von dort aus kein Weg zurück führt.
Zustand 4 ist mit Zustand 6 verbunden.
Damit sind alle Zustände eingeordnet und die Zerlegung der Zustände ist:
\[
S_{/\lr} = \{\{1\}, \{2,3,6\}, \{4, 5\}\}
\]
\end{example}

Die \link{def:mk-erreichbar}{Erreichbarkeit von Zuständen} können wir auf die
Äquivalenzklassen übertragen:
\begin{definition}{Ordnung von Zustandsklassen}{mk-ord}
Sei $S$ der Zustandsraum einer Markovkette, der durch die Verbundenheitsrelation
$\lr$ in Klassen $S_{/\lr}=\{G_1, G_2,...,G_n\}$ zerlegt wird. Auf diesen
Klassen definieren wir folgende Relation:
\[
\preceq\ =\{(G_i, G_j)\ |\ \exists s_i\in G_i, s_j\in G_j: s_i\to s_j\}\ \subseteq\ S_{/\lr}\times S_{/\lr}
\]
Die Relation bildet eine partielle Ordnung\more{mfnf-o-rel}, ist also reflexiv, anti-symmetrisch und transitiv. Für
$G_i\preceq G_j$ schreiben wir auch $G_i\to G_j$.
\end{definition}

Die so definierte Ordnung ist jedoch im Allgemeinen nicht \emph{total}, da nicht
für jedes Paar von Zustandsklassen definiert sein muss, ob $G_i\preceq G_j$ oder
$G_j \preceq G_i$ gilt. Das ist zum Beispiel der Fall, wenn die Klassen nicht
miteinander verbunden sind, also im Interaktionsgraphen unverbundene Subgraphen
bilden oder wenn zwei unverbundene Klassen in einem dritten Zustand
"`zusammenlaufen"'.

\begin{definition}{Abgeschlossene Klasse}{mk-abg}
Sei $G$ eine Zustandsklasse einer Markovkette. $G$ heißt \defw{abgeschlossen},
wenn es keine Klasse gibt, die von $G$ erreicht werden kann:
\[
\nexists H\ne G: G\preceq H
\]
\end{definition}

Abgeschlossene Klassen werden zusätzlich nach der Anzahl ihrer Zustände
unterschieden:

\begin{definition}{Absorbierende Klasse/Markovkette}{mk-abs}
Sei $(X)_{n\in N_0}$ eine Markovkette. Ist $G$ eine \link{def:mk-abg}{abgeschlossene Klasse}
mit $|G|=1$, so heißt die Klasse $G$ \defw{absorbierend}.

Sind alle abgeschlossenen Klassen absorbierend, so heißt die
Markovkette \defw{absorbierend}.
\end{definition}

Sind alle Zustände einer Markovkette miteinander verbunden, gibt es nur eine
Äquivalenzklasse:
\begin{definition}{Irreduzibilität}{mk-irred}
Sei $(X)_{n\in N_0}$ eine Markovkette mit Zustandsraum $S$, der durch die
Verbundenheitsrelation $\lr$ in die Menge $S_{/\lr}$ zerlegt wird. Existiert nur
eine einzige Klasse, das heißt es gilt
\[
S_{/\lr}\ = \{S\}
\]
heißt die Markovkette \defw{irreduzibel}.
\end{definition}

Das ist genau der Fall, dass alle Zustände mit allen anderen Zuständen verbunden
sind, sodass genau eine Klasse besteht.

\begin{lemma}
Die Klasse $S$ einer irreduziblen Markovkette ist abgeschlossen.
\end{lemma}
 
\subsection{Rückkehreigenschaften}

In diesem Abschnitt werden Zustände von Markovketten bezüglich ihres
Rückkehrverhaltens untersucht\more{wiki-mk-properties}.

\begin{definition}{Rückkehrzeit}{mk-rueckk}
Sei $(X)_{n\in N_0}$ eine Markovkette mit Zustandsraum $S$ und $y\in S$
\link{def:mk-rekurr}{rekurrent} und der Startzustand $X_0$ der Markovkette.
Dann heißt
\[
T_y = min\{n>0:\ X_n = y\}
\]
die (zufällige) \defw{Rückkehrzeit} in den Zustand $y$. Zusätzlich bezeichnet
\[
M_y = \{k>0: P(T_y=k|X_0=y > 0\}
\]
die Menge der Zeitschritte, in denen ausgehend von $y$ wieder $y$ erreicht
werden kann.
\end{definition}

\begin{definition}{Periode eines Zustands}{mk-periode}
Sei $(X)_{n\in N_0}$ eine Markovkette mit Zustandsraum $S$ und $y\in S$ ein
Zustand, der \emph{nur} in $M_y=\{d, 2d, 3d, ...\}$ Schritten erreicht werden
kann, so heißt
\[
d = \mathrm{ggT}(M_y)
\]
die \defw{Periode} des Zustands $y$.
\end{definition}

\begin{definition}{Aperiodischer Zustand, aperiodische Markovkette}{mk-aperiod}
Sei $y$ ein Zustand einer Markovkette. Besitzt dieser Zustand die Periode 1, so
heißt $y$ \defw{aperiodisch}.

Sind alle Zustände einer Markovkette aperiodisch, heißt die Markovkette
\defw{aperiodisch}.
\end{definition}

\begin{definition}{Ergodische Markovkette}{mk-ergod}
Ist eine Markovkette mit endlich vielen Zuständen \link{def:mk-irred}{irreduzibel} und
\link{def:mk-aperiod}{aperiodisch}, so heißt sie \defw{ergodisch}.
\end{definition}

Die Zustände eines \link{def:stochp}{stochastischen Prozess} können danach
unterschieden werden, ob man in jedem Fall zu dem Zustand, in dem man gestartet
ist, zurückkehren kann (rekurrent) oder nicht (transient)
\more{brilliant-transience-recurrence}:

\begin{definition}{Rekurrenter Zustand}{mk-rekurr}
Sei $y$ ein Zustand einer Markovkette. Dann heißt $y$ \defw{rekurrent}, wenn
es für jeden von $y$ aus \link{def:mk-erreichbar}{erreichbaren} Zustand einen
Weg zurück zu $i$ gibt.
\end{definition}

\begin{definition}{Transienter Zustand}{mk-trans}
Sei $y$ ein Zustand einer Markovkette. Dann heißt $y$ \defw{transient}, wenn $y$
nicht rekurrent ist.
\end{definition}

Per Definition besteht bei transienten Zuständen die "`Gefahr"', nicht mehr in
diese zurückkehren zu können. Auf lange Sicht sind transiente Zustände daher nur
Zwischenzustände.

Die Eigenschaften \link{def:mk-rekurr}{Rekurrenz} und
\link{def:mk-trans}{Transienz} sind Klasseneigenschaften: Besitzt ein Zustand
eine dieser Eigenschaften, so auch alle anderen Zustände der Klasse. (Innerhalb
einer Klasse sind schließlich immer alle Zustände miteinander verbunden, es gibt
also immer einen Weg zurück zum Startzustand. Damit "`überträgt"' sich die
Transienz bzw. Rekurrenz eines Zustands auf die gesamte Klasse.)

\begin{lemma}
Jede \link{def:mk-abg}{abgeschlossene} Klasse ist rekurrent.

Begründung: Eine abgeschlossene Klasse kann nur betreten, aber nicht mehr
verlassen werden. Somit sind nur Zustände der Klasse erreichbar, sodass immer
zum Startzustand zurückgekehrt werden kann.
\end{lemma}

\begin{example}{Transiente/Rekurrente Zustandsklassen}{mk-tr-rek}
Gegeben sei eine Markovkette mit folgendem Interaktionsgraph:

\begin{tikzpicture}
  \node[v] (1) at (2,1) {1};
  \node[v] (2) at (4,2) {2};
  \node[v] (3) at (6,2) {3};
  \node[v] (6) at (5,0) {6};
  \node[v] (4) at (8,1) {4};
  \node[v] (5) at (10,1) {5};

  \draw[e] (1) to[loop above] ();
  \draw[e] (6) -- (1);
  \draw[e] (6) -- (3);
  \draw[e] (2) to[bend right] (3);
  \draw[e] (3) to[bend right] (2);
  \draw[e] (2) to[loop above] ();
  \draw[e] (2) -- (6);
  \draw[e] (6) -- (4);
  \draw[e] (4) to[bend left] (5);
  \draw[e] (5) to[bend left] (4);
\end{tikzpicture}

Die Markovkette besitzt die Zustandsklassen $S_{1}=\{1\}$, $S_{2,3,6} =\{2,3,6\}$
und $S_{4,5}=\{4, 5\}$. Zustand 1 ist rekurrent, da alle erreichbaren Zustände
(also nur der Zustand selbst) wieder einen Pfad zurück zum Zustand 1 besitzen.
Wenn man in 1 startet, kann es also nicht passieren, dass man nicht mehr dahin
zurückkommt.
Die Klasse $S_{2,3,6}$ ist transient, da man von dort aus auch in die Klassen
$S_1$ und $S_{4,5}$ kommen kann, von denen jedoch kein Pfad zurück in die
"`Startklasse"' führt. Die Klasse $S_{4,5}$ ist rekurrent, da sie (wie auch
$S_1$) nicht verlassen werden kann.
\end{example}

    \section{Langzeitverhalten}
    \newcommand{\stat}{\underline{\pi}} % stationäre Wahrscheinlichkeit

Dieser Abschnitt beschäftigt sich mit der \link{def:mk-vert}{Verteilung der
Markovkette} nach unendlich vielen Schritten.

\textbf{Achtung: Die hier verwendete
Notation weicht zum Teil deutlich von den Vorlesungsvideos ab.}

\begin{definition}{Stationäre Verteilung}{mk-stat}
Sei $(X)_{n\in N_0}$ eine Markovkette mit \link{def:mk-matr}{Übergangsmatrix}
$\Pi$ und endlichem Zustandsraum $S$. Dann heißt die
\link{def:mk-vert}{Verteilung} $\stat$ \defw{stationäre Verteilung},
wenn gilt:
\[
\stat\cdot\Pi = \stat
\]
\end{definition}

Gelangt oder startet man in einer stationären Verteilung, ändert sich die
Wahrscheinlichkeit, sich in einem Zustand zu befinden, nicht mehr.

\subsection{Ergodische Markovketten}

\begin{theorem}{Hauptsatz für ergodische Markovketten}{ergod}
Sei $(X)_{n\in N_0}$ eine \link{def:mk-ergod}{ergodische} Markovkette mit
\link{def:mk-matr}{Übergangsmatrix} $\Pi$ und $y$ ein Zustand der Markovkette
mit \link{def:mk-rueckk}{Rückkehrzeit} $T_y$. Dann
besitzt die Markovkette \emph{genau} eine stationäre Verteilung $\stat$
für die gilt:
\[
\lim_{n\to\infty}\Pi(X_n = y) = lim_{n\to\infty}\Pi^n(x, y) = \stat(y)
\]
Weiterhin existiert ein Zusammenhang zwischen der erwarteten Rückkehrzeit und
der stationären Verteilung:
\[
\stat(y) = \frac{1}{\E(T_y|X_0=y)}
\]
\end{theorem}

Je weiter eine ergodische Markovkette läuft, desto mehr nähert sich die
Wahrscheinlichkeitsverteilung an die stationäre Verteilung an. Die stationäre
Verteilung gibt – zumindest auf lange Sicht – an, mit welcher Wahrscheinlichkeit
man sich in den jeweiligen Zuständen befindet.

Die stationäre Verteilung einer ergodischen Markovkette kann berechnet
werden. Dafür nutzt man die Tatsache, dass die stationäre Verteilung (per
Definition) ein Eigenvektor\more{mfnf-ew-ev} der Übergangsmatrix. Zusätzlich
muss wird eine Gleichung eingefügt, die die Normierung der stationären
Verteilung auf 1 sicherstellt. Es ergibt sich folgendes
lineares Gleichungssystem:
\begin{equation}\label{eq:stat-lgs}
\begin{pmatrix}
\Pi_{00}-1 & \Pi_{10} & \ldots & \Pi_{j0} \\
\Pi_{01} & \Pi_{11}-1 &        & \Pi_{j1} \\
\vdots   &            & \ddots & \vdots \\
\Pi_{0i} & \Pi_{1i} & \ldots &   \Pi_{ji}-1 \\
    1    &     1    & \ldots &   1
\end{pmatrix}\cdot \vec{\stat} = \begin{pmatrix}0\\0\\\vdots\\0\\1\end{pmatrix}
\end{equation}
Wichtig ist auch, dass die Matrix $\Pi$ transponiert aufgeschrieben, da eine
Spalte der Matrix einer Gleichung entspricht.

\subsection{Nichtergodische Markovketten}

\newcommand{\lzv}{\Pi^\infty} % Langzeitverhalten

Bei Nichtergodischen Markovketten ist das Langzeitverhalten nicht so einfach
bestimmbar, da die stationäre Verteilung vom Startzustand abhängt. Das
Langzeitverhalten der Markovkette wird durch die Matrix $\lzv$ beschrieben,
die für jeden möglichen Startzustand die entsprechende Langzeitverteilung
angibt.

\medskip
Sei $(X)_{n\in N_0}$ eine Markovkette mit endlichem Zustandsram $S$ und
\link{def:mk-matr}{Übergangsmatrix} $\Pi$, wobei $T\subseteq S$ die Menge aller
\link{def:mk-trans}{transienten} Zustände ist. Es gelten die nachfolgende Regeln
zur Bestimmung des Langzeitverhaltens.

In einer \link{def:mk-abg}{abgeschlossenen} Klasse $C$ kann das
Langzeitverhalten als \link{def:mk-stat}{stationäre Verteilung}
$\stat^{(C)}$ der auf die Zustände in $C$ beschränkten Markovkette bestimmt
werden (vergleiche \eqref{eq:stat-lgs}):
\begin{equation}\label{eq:lzv-abg-abg}
  \forall x,y\in C: \lzv(x,y) = \stat^{(C)}(y)
\end{equation}
Da eine \link{def:mk-abg}{abgeschlossene} Klasse $C$ nicht
verlassen werden kann, gilt für $x\in C$ und $y \notin C$:
\begin{equation}\label{eq:lzv-abg-x}
  \lzv(x,y) = 0
\end{equation}
Damit ist das Langzeitverhalten der Startzustände in abgeschlossenen Klassen
bestimmt.

\medskip
Das Langzeitverhalten bei Start in einem \link{def:mk-trans}{transienten}
Zustand $t\in T$ muss die
Wahrscheinlichkeit der Absorption durch eine abgeschlossene Klasse $C$
berücksichtigt werden, die mit $p_{abs}(t,C)$ bezeichnet wird. Die
Absorptionswahrscheinlichkeit berechnet sich rekursiv als Wahrscheinlichkeit des
direkten Übergangs in die absorbierende Klasse plus die Wahrscheinlichkeit des
indirekten Übergangs mit einem zusätzlichen Schritt innerhalb der transienten
Klasse:
\[
p_{abs,t}(C) = \sum_{y\in C} \Pi(x,y) + \sum_{z\in T}\Pi(x,z)\cdot p_{abs,z} \\
\]
Für alle $t$ einer transienten Klasse entsteht so ein Gleichungssystem.

Die Wahrscheinlichkteit, bei Start in $t$ in $y\in S$ zu
enden, berechnet sich durch die Absorptionswahrscheinlichkeit verknüpft mit
der stationären Wahrscheinlichkeit $\stat^{(C)}$ innerhalb von $C$:
\begin{equation}\label{eq:lzv-trans-abg}
\lzv(x,y) = p_{abs,t}(C)\cdot\stat^{(C)}(y)
\end{equation}
Die Wahrscheinlichkeit, sich in einem \link{def:mk-trans}{transienten}
Zustand zu befinden, sinkt auf lange Sicht auf 0:
\begin{equation}\label{eq:lzv-trans-trans}
\forall x\in S, y\in T: \lzv(x, y) = 0
\end{equation}
Damit ist das Langzeitverhalten für den Start in einem transitiven Zustand
definiert.
Da in endlichen Markovketten alle abgeschlossenen Klassen rekurrent sind
\warn{Begründung bzw. Beweis fehlt}, sind
durch die obigen Definitionen das Langzeitverhalten bestimmt.

\begin{example}{Langzeitverhalten einer nichtergodischen Markovkette}{mk-lzv-nerg}
Gegeben sei folgende Markovkette:

\begin{minipage}{0.45\textwidth}
  \begin{tikzpicture}
    \node[v] (1) at (0,1.5) {1};
    \node[v] (2) at (-1.5,0) {2};
    \node[v] (3) at (0,0) {3};
    \node[v] (4) at (1.5,0) {4};

    \draw[e] (1) to[loop above] ();
    \draw[e] (2) to[loop below] ();
    \draw[e] (3) to[loop below] ();
    \draw[e] (4) to[loop below] ();
    \draw[e] (1) -- (2);
    \draw[e] (1) -- (3);
    \draw[e] (1) -- (4);
    \draw[e] (2) to[bend left] (3);
    \draw[e] (3) to[bend left] (2);
  \end{tikzpicture}
\end{minipage}
\begin{minipage}{0.45\textwidth}
  \[\Pi = \begin{pmatrix}
    0.1 & 0.4 & 0.3 & 0.2 \\
     0  & 0.2 & 0.8 &  0  \\
     0  & 0.9 & 0.1 &  0  \\
     0  &  0  &  0  &  1
  \end{pmatrix}\]
\end{minipage}

Die Markovkette ist nicht \link{def:mk-irred}{irreduzibel}, da mehr als eine
Klasse existiert ($S_{/\lr} = {S_1, S_{2,3}, S_{4}}$), Satz \ref{satz:ergod}
kann also nicht angewandt werden.

\medskip
Da Klasse $S_4$ abgeschlossen und Zustand 4 absorbierend ist, ergibt sich als
Sonderfall von \eqref{eq:lzv-abg-abg} und \eqref{eq:lzv-abg-x} folgende
Verteilung:
\[\begin{pmatrix}0 & 0 & 0 & 1\end{pmatrix}\]

Klasse $S_{2,3}$ ist abgeschlossen. Die stationäre Verteilung $\stat^{(S_{2,3})}$
ergibt sich als Lösung des Gleichungssystems (Vergleiche \eqref{eq:stat-lgs})
und bestimmt $\lzv(2..3, 2..3)$:
\[
\begin{pmatrix}
0.2 -1 & 0.9     \\
0.8    & 0.1 - 1 \\
   1   &  1
\end{pmatrix}\cdot \vec{\stat}^{(S_{2,3})} = \begin{pmatrix}0\\0\\1\end{pmatrix}
\iff \vec{\stat}^{(S_{2,3})} =
\renewcommand\arraystretch{1.3}
\begin{pmatrix}0.53\\0.47\end{pmatrix}
\]

Zustand 1 ist transient, damit gilt $\lzv(1,1) = 0$
(\eqref{eq:lzv-trans-trans}). Als Zwischenergebnis bestimmen wir die
Absorptionswahrscheinlichkeiten $p_{abs,1}(S_{2,3})$ und $p_{abs,1}(S_{4})$:
\begin{align*}
p_{abs,1}(S_{2,3}) &=&  0.4 + 0.3 + 0.1 \cdot p_{abs,1}(S_{2,3}) & \implies
p_{abs,1}(S_{2,3}) = \frac{0.7}{0.9} = 0.78  \\
p_{abs,1}(S_{4})   &=&  0.2 + 0.1 \cdot p_{abs,1}(S_4) &\implies
p_{abs,1}(S_4) =\frac{0.2}{0.9} = 0.22
\end{align*}
Durch Anwendung von Gleichung \eqref{eq:lzv-trans-abg} ergibt sich:
\begin{align*}
\lzv(1,2) &=& p_{abs,1}(S_{2,3}) &\cdot\stat^{(S_{2,3})}(2) &= 0.41 \\
\lzv(1,3) &=& p_{abs,1}(S_{2,3}) &\cdot\stat^{(S_{2,3})}(3) &= 0.37 \\
\lzv(1,4) &=& p_{abs,1}(S_4)     &\cdot\stat^{(S_4)}    (4) &= 0.22
\end{align*}
Damit sind alle Einträge der Matrix $\lzv$ bestimmt:
\[
\lzv = \begin{pmatrix}
   0  & 0.41 & 0.37 & 0.22 \\
   0  & 0.53 & 0.47 &  0   \\
   0  & 0.53 & 0.47 &  0   \\
   0  &  0   &  0   &  1
\end{pmatrix}
\]
Als Probe kann sichergestellt werden, dass die Matrix stochastisch ist, sich
also die Einträge jeder Zeile zu 1 summieren.
\end{example}

\subsection{Reversibilität}

\begin{definition}{Reversibilität; Detaillierte Balance}{mk-rev}
Sei $(X)_{n\in N_0}$ eine Markovkette mit Zustandsraum $S$, Übergangsmatrix
$\Pi$ und $\pi$ eine \link{def:mk-vert}{Verteilung}. Die Markovkette heißt
\defw{reversibel}, wenn gilt:
\[
\forall x,y\in S:\ \pi(y)\cdot\Pi(y,x) = \pi(x)\cdot\Pi(x,y)
\]
Diese Bedingung wird auch als \defw{detaillierte Balance} bzw. als
\defw{detailliertes Gleichgewicht} bezeichnet.
\end{definition}

In einer reversiblen Markovkette kann also nicht unterschieden werden, ob der
Prozess vorwärts oder rückwärts abläuft.

\begin{theorem}{}{mk-db}
Sei $\pi$ eine Verteilung, die die Bedingung der detaillierten Balance
für eine Markovkette erfüllt. Dann ist $\pi$ eine
\link{def:mk-stat}{stationäre Verteilung}.
\end{theorem}

Reversibilität ist eine stärkere Bedinung als Stationarität, es gibt also
stationäre Verteilungen, die nicht die Bedingung der detaillierten Balance
erfüllen.

\begin{theorem}{Schnittprinzip}{mk-schnittp}
Sei $(X)_{n\in N_0}$ eine \link{def:mk-ergod}{ergodische} Markovkette. Ist der
\link{def:mk-igraph}{Interaktionsgraph} von $(X)$ linear, das jeder Schnitt
zwischen zwei verbundenen Zuständen zerlegt den Graph in zwei disjunkte und
unverbundene Teile, so ist die stationäre Verteilung immer reversibel.
\end{theorem}

Da wir wissen, dass eine ergodische Markovkette immer genau eine stationäre
Verteilung besitzt (Satz \ref{satz:ergod}), folgt aus dem Schnittprinzip, dass
diese Verteilung auch reversibel ist. Dann können wir für die Verteilung die
detaillierten Balance annehmen. Das ist besonders hilfreich, wenn der
betrachtete Prozess sehr viele Zustände besitzt und die stationäre Verteilung
aufwändig zu berechnen ist.

\begin{example}{Warteschlange}{}
Ein bestimmter Server kann bis zu 100 Anfragen parallel bearbeiten. Pro Sekunde
erhält der Server mit Wahrscheinlichkeit $p^+ = 0.2$ eine neue Anfrage und schließt
mit Wahrscheinlichkeit $p^-=0.21$ eine Anfrage ab. Werden bereits 100 Anfragen
bearbeitet, werden neue Anfragen abgewiesen.

\emph{Wie hoch ist die Wahrscheinlichkeit, dass ein ankommender Auftrag abgewiesen
wird? Wie hoch ist die Wahrscheinlichkeit, dass die Maschine leerläuft?}

Die Anzahl der aktiven Anfragen kann durch eine Markovkette mit Zustandsraum
$S=\{0, 1,\ldots, 100\}$ beschrieben werden:

\medskip
\begin{tikzpicture}
  \node[v] (0) at (0,1) {0};
  \node[v] (1) at (2,1) {1};
  \node[v] (2) at (4,1) {2};
  \node[v] (n) at (6, 1) {$\ldots$};
  \node[v] (99) at (8,1) {99};
  \node[v] (100) at (10,1) {100};

  \draw[e] (0) to[loop left] node[anchor=south] {$p^-$} ();
  \draw[e] (100) to[loop right] node[anchor=south] {$p^+$} ();
  \draw[e] (0) to[bend left] node[anchor=south] {$p^+$} (1);
  \draw[e] (1) to[bend left] node[anchor=north] {$p^-$} (0);
  \draw[e] (1) to[bend left] node[anchor=south] {$p^+$} (2);
  \draw[e] (2) to[bend left] node[anchor=north] {$p^-$} (1);
  \draw[e] (2) to[bend left] node[anchor=south] {$p^+$} (n);
  \draw[e] (n) to[bend left] node[anchor=north] {$p^-$} (2);
  \draw[e] (n) to[bend left] node[anchor=south] {$p^+$} (99);
  \draw[e] (99) to[bend left] node[anchor=north] {$p^-$} (n);
  \draw[e] (99) to[bend left] node[anchor=south] {$p^+$} (100);
  \draw[e] (100) to[bend left] node[anchor=north] {$p^-$} (99);
\end{tikzpicture}

Die Markovkette besteht nur aus einer einzigen Klasse, ist daher abgeschlossen,
rekurrent und irreduzibel. Da die Zustände 0 und 100 Periode 1 besitzten, ist
die Markovkette aperiodisch. Folglich ist die Markovkette auch ergodisch, es
existiert also eine eindeutige stationäre Verteilung $\stat$.

Diese Verteilung ist wegen der 101 Zustände sehr aufwändig zu berechnen. Wir
können jedoch das Schnittprinzip (Satz \ref{satz:mk-schnittp}) anwenden, da der
Interaktionsgraph der Markovkette linear ist. Das bedeutet, dass für die
stationäre Wahrscheinlichkeit die Bedingung der detaillierten Balance gilt:
\begin{align*}
\forall x,y\in S:\ &\pi(y)\cdot\Pi(y,x) = \pi(x)\cdot\Pi(x,y) \\
\implies&\stat(i)\cdot p^+ = \stat(i+1)\cdot p^- \\
\implies&\stat(i+1) = \frac{p^+}{p^-}\cdot\stat(i) \\
\implies&\stat = (c, c\cdot\gamma,c\cdot\gamma^2,\ldots,
c\cdot\gamma^{100})^{\T},\ \gamma = \frac{p^+}{p^-}, c\in\R
\end{align*}
Da $\stat$ eine Verteilung ist, wissen wir dass die Einträge insgesamt 1
ergeben. Damit kann $c$ bestimmt werden:
\begin{align*}
1 &= \sum_{i=0}^{100}c\cdot\gamma^i = c \sum_{i=0}^{100}\gamma^i \\
\overset{\text{Geom. Summenformel}}{\iff} 1 &= c\cdot\frac{1-\gamma^{101}}{1-\gamma} \\
\implies \stat(i) &= c\cdot\gamma^i = \frac{1-\gamma}{1-\gamma^{101}}\cdot \gamma^i
\end{align*}

Die Wahrscheinlichkeit, dass ein ankommender Auftrag abgewiesen wird, ist
$\stat(100)=3.6\cdot10^{-4}$, der Server also bereits voll ausgelastet ist. Die
Wahrscheinlichkeit für Leerlauf ist $\stat(0) = 4.79\cdot10^{-2}$.
\end{example}

    \section{Markovketten mit unendlichem Zustandsraum}
    Bisher haben wir endliche Markovketten betrachtet. Markovketten können jedoch
auch einen unendlich großen Zustandsraum wie $S=\N$ besitzen (der Zustandsraum
muss jedoch abzählbar unendlich sein). Dann können wir wie zuvor auch
\begin{itemize}
\item die Übergangsmatrix $\Pi$ aufstellen,
\item Äquivalenzklassen betrachten,
\item Interaktionsgraphen untersuchen,
\item Zustände bezüglich ihrer Periode untersuchen.
\end{itemize}

Allerdings ist das Rückkehrverhalten und die stationäre Verteilung bei
ergodischen unendlichen Markovketten anders. Wir definieren daher die Begriffe
\link{def:mk-rekurr}{rekurrent} und \link{def:mk-trans}{transient} neu und
unterscheiden unendliche und endliche Rückkehrzeit bei rekurrenten Zuständen.

\begin{definition}{Rekurrent, null-rekurrent, transient}{mk-inf-rekurr}
Sei $(X)_{n\in N_0}$ eine Markovkette mit abzählbar unendlich großem
Zustandsraum $S$. Dann heißt ein Zustand $y\in S$ \defw{rekurrent}, falls die
erwartete Rückkehrzeit in den Zustand endlich ist.

$y$ heißt \defw{null-rekurrent}, falls die Rückkehrzeit unendlich, die Rückkehr
aber sicher ist, d.h mit Wahrscheinlichkeit 1 stattfindet.

Zustand $y$ heißt \defw{transient}, falls mit positiver Wahrscheinlihckeit keine
Rückkehr zu $y$ stattfindet.
\end{definition}

\warn{Hier ist nicht klar, ob diese Definitionen äquivalent zu den bisherigen
Definitionen sind, oder ausschließlich für unendliche Markovketten gelten.}


  \printbibliography
\end{document}
