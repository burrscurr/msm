\newcommand{\stat}{\underline{\pi}} % stationäre Wahrscheinlichkeit

Dieser Abschnitt beschäftigt sich mit der \link{def:mk-vert}{Verteilung der
Markovkette} nach unendlich vielen Schritten.

\textbf{Achtung: Die hier verwendete
Notation weicht zum Teil deutlich von den Vorlesungsvideos ab.}

\begin{definition}{Stationäre Verteilung}{mk-stat}
Sei $(X)_{n\in N_0}$ eine Markovkette mit \link{def:mk-matr}{Übergangsmatrix}
$\Pi$ und endlichem Zustandsraum $S$. Dann heißt die
\link{def:mk-vert}{Verteilung} $\stat$ \defw{stationäre Verteilung},
wenn gilt:
\[
\stat\cdot\Pi = \stat
\]
\end{definition}

Gelangt oder startet man in einer stationären Verteilung, ändert sich die
Wahrscheinlichkeit, sich in einem Zustand zu befinden, nicht mehr.

\subsection{Ergodische Markovketten}

\begin{theorem}{Hauptsatz für ergodische Markovketten}{ergod}
Sei $(X)_{n\in N_0}$ eine \link{def:mk-ergod}{ergodische} Markovkette mit
\link{def:mk-matr}{Übergangsmatrix} $\Pi$ und $y$ ein Zustand der Markovkette
mit \link{def:mk-rueckk}{Rückkehrzeit} $T_y$. Dann
besitzt die Markovkette \emph{genau} eine stationäre Verteilung $\stat$
für die gilt:
\[
\lim_{n\to\infty}\Pi(X_n = y) = lim_{n\to\infty}\Pi^n(x, y) = \stat(y)
\]
Weiterhin existiert ein Zusammenhang zwischen der erwarteten Rückkehrzeit und
der stationären Verteilung:
\[
\stat(y) = \frac{1}{\E(T_y|X_0=y)}
\]
\end{theorem}

Je weiter eine ergodische Markovkette läuft, desto mehr nähert sich die
Wahrscheinlichkeitsverteilung an die stationäre Verteilung an. Die stationäre
Verteilung gibt – zumindest auf lange Sicht – an, mit welcher Wahrscheinlichkeit
man sich in den jeweiligen Zuständen befindet.

Die stationäre Verteilung einer ergodischen Markovkette kann berechnet
werden. Dafür nutzt man die Tatsache, dass die stationäre Verteilung (per
Definition) ein Eigenvektor\more{mfnf-ew-ev} der Übergangsmatrix. Zusätzlich
muss wird eine Gleichung eingefügt, die die Normierung der stationären
Verteilung auf 1 sicherstellt. Es ergibt sich folgendes
lineares Gleichungssystem:
\begin{equation}\label{eq:stat-lgs}
\begin{pmatrix}
\Pi_{00}-1 & \Pi_{10} & \ldots & \Pi_{j0} \\
\Pi_{01} & \Pi_{11}-1 &        & \Pi_{j1} \\
\vdots   &            & \ddots & \vdots \\
\Pi_{0i} & \Pi_{1i} & \ldots &   \Pi_{ji}-1 \\
    1    &     1    & \ldots &   1
\end{pmatrix}\cdot \vec{\stat} = \begin{pmatrix}0\\0\\\vdots\\0\\1\end{pmatrix}
\end{equation}
Wichtig ist auch, dass die Matrix $\Pi$ transponiert aufgeschrieben, da eine
Spalte der Matrix einer Gleichung entspricht.

\subsection{Nichtergodische Markovketten}

\newcommand{\lzv}{\Pi^\infty} % Langzeitverhalten

Das Langzeitverhalten von nichtergodischen Markovketten hängt von Startzustand
ab. Es wird in der Matrix $\lzv$ zusammengefasst, die jedem Startzustand die auf
lange Sicht zu erwartende Verteilung zuordnet. Die Regeln zur Bestimmung des
Langzeitverhaltens unterscheiden sich je nach dem ob der Startzustand in
einer \link{def:mk-abg}{abgeschlossenen} oder \link{def:mk-trans}{transienten}
Klasse liegt.

\medskip
Im Folgenden sei $(X)_{n\in N_0}$ eine Markovkette mit endlichem Zustandsram $S$
und \link{def:mk-matr}{Übergangsmatrix} $\Pi$.

Das Langzeitverhalten bei Start in einem Zustand einer
\link{def:mk-abg}{abgeschlossenen} Klasse $C$ entspricht der
\link{def:mk-stat}{stationäre Verteilung} $\stat^{(C)}$ der auf die Zustände in
$C$ beschränkten Markovkette (siehe \eqref{eq:stat-lgs}). Der Übergang in einen
Zustand außerhalb der Klasse $C$ ist ausgeschlossen. Für $x, x' \in C$ und
$y\notin C$ gilt:
\begin{align}\label{eq:lzv-abg-abg}
  \lzv(x,x') &= \stat^{(C)}(x') \\
  \label{eq:lzv-abg-x}
  \lzv(x,y) &= 0
\end{align}

\medskip

Da \link{def:mk-trans}{transiente} Klassen nur verlassen, aber nicht in sie
zurückgekehrt werden kann, ist die langfristige Wahrscheinlichkeit, sich in
einem transienten Zustand zu befinden, null. Die Wahrscheinlichkeit, bei Start
in einem transienten Zustand $t$ langfristig in die abgeschlossene Klasse $C$
überzugehen, wird mit $p_{t\to C}$ bezeichnet \warn{andere Schreibweise als in
der Vorlesung}. Diese sogenannte Absorptionswahrscheinlichkeit verteilt sich
dann innerhalb von $C$ gemäß der berechneten stationären Verteilung.

Seien $x,x'$ transiente Zustände und $y$ Zustand einer
abgeschlossenen Klasse $C$. Dann gilt:
\begin{align}
  \label{eq:lzv-trans-abg}
  \lzv(x,y) &= p_{t\to C}\cdot\stat^{(C)}(y) \\
  \label{eq:lzv-trans-trans}
  \lzv(x,x') &= 0
\end{align}

Die Absorptionswahrscheinlichkeiten von Zuständen einer transienter Klasse $T$
bezüglich des Übergangs in die abgeschlossene Klasse $C$ bilden ein
lineares Gleichungssystem. Jede Gleichung für $t\in T$ ergibt sich
aus der Wahrscheinlichkeit des direkten Überangs in $C$ und der
Wahrscheinlichkeit, indirekt über die anderen Zustände von $T$ in $C$
überzugehen:
\[
p_{t\to C} = \sum_{c\in C} \Pi(t,c) + \sum_{t'\in T}\Pi(t,t')\cdot p_{t'\to C}
\]

\begin{example}{Langzeitverhalten einer nichtergodischen Markovkette}{mk-lzv-nerg}
Gegeben sei folgende Markovkette:

\begin{minipage}{0.45\textwidth}
  \begin{tikzpicture}
    \node[v] (1) at (0,1.5) {1};
    \node[v] (2) at (-1.5,0) {2};
    \node[v] (3) at (0,0) {3};
    \node[v] (4) at (1.5,0) {4};

    \draw[e] (1) to[loop above] ();
    \draw[e] (2) to[loop below] ();
    \draw[e] (3) to[loop below] ();
    \draw[e] (4) to[loop below] ();
    \draw[e] (1) -- (2);
    \draw[e] (1) -- (3);
    \draw[e] (1) -- (4);
    \draw[e] (2) to[bend left] (3);
    \draw[e] (3) to[bend left] (2);
  \end{tikzpicture}
\end{minipage}
\begin{minipage}{0.45\textwidth}
  \[\Pi = \begin{pmatrix}
    0.1 & 0.4 & 0.3 & 0.2 \\
     0  & 0.2 & 0.8 &  0  \\
     0  & 0.9 & 0.1 &  0  \\
     0  &  0  &  0  &  1
  \end{pmatrix}\]
\end{minipage}

Die Markovkette ist nicht \link{def:mk-irred}{irreduzibel}, da mehr als eine
Klasse existiert ($S_{/\lr} = {S_1, S_{2,3}, S_{4}}$), Satz \ref{satz:ergod}
kann also nicht angewandt werden.

\medskip
Da Klasse $S_4$ abgeschlossen und Zustand 4 absorbierend ist, ergibt sich als
Sonderfall von \eqref{eq:lzv-abg-abg} und \eqref{eq:lzv-abg-x} folgende
Verteilung:
\[\begin{pmatrix}0 & 0 & 0 & 1\end{pmatrix}\]

Klasse $S_{2,3}$ ist abgeschlossen. Die stationäre Verteilung $\stat^{(S_{2,3})}$
ergibt sich als Lösung des Gleichungssystems (Vergleiche \eqref{eq:stat-lgs})
und bestimmt $\lzv(2..3, 2..3)$:
\[
\begin{pmatrix}
0.2 -1 & 0.9     \\
0.8    & 0.1 - 1 \\
   1   &  1
\end{pmatrix}\cdot \vec{\stat}^{(S_{2,3})} = \begin{pmatrix}0\\0\\1\end{pmatrix}
\iff \vec{\stat}^{(S_{2,3})} =
\renewcommand\arraystretch{1.3}
\begin{pmatrix}0.53\\0.47\end{pmatrix}
\]

Zustand 1 ist transient, damit gilt nach Gleichung \eqref{eq:lzv-trans-trans}
$\lzv(1,1) = 0$. Als Zwischenergebnis bestimmen wir die
Absorptionswahrscheinlichkeiten $p_{1\to S_{2,3}}$ und $p_{1\to S_{4}}$:
\begin{align*}
p_{1\to S_{2,3}} &=&  0.4 + 0.3 + 0.1 \cdot p_{1\to S_{2,3}}& \implies
p_{1\to S_{2,3}} = \frac{0.7}{0.9} = 0.78  \\
p_{1\to S_{4}}   &=&  0.2 + 0.1 \cdot p_{1\to S_4} &\implies
p_{1\to S_4} =\frac{0.2}{0.9} = 0.22
\end{align*}
Durch Anwendung von Gleichung \eqref{eq:lzv-trans-abg} ergibt sich:
\begin{align*}
\lzv(1,2) &=& p_{1\to S_{2,3}} &\cdot\stat^{(S_{2,3})}(2) &= 0.41 \\
\lzv(1,3) &=& p_{1\to S_{2,3}} &\cdot\stat^{(S_{2,3})}(3) &= 0.37 \\
\lzv(1,4) &=& p_{1\to S_4}     &\cdot\stat^{(S_4)}    (4) &= 0.22
\end{align*}
Damit sind alle Einträge der Matrix $\lzv$ bestimmt:
\[
\lzv = \begin{pmatrix}
   0  & 0.41 & 0.37 & 0.22 \\
   0  & 0.53 & 0.47 &  0   \\
   0  & 0.53 & 0.47 &  0   \\
   0  &  0   &  0   &  1
\end{pmatrix}
\]
Als Probe kann sichergestellt werden, dass die Matrix stochastisch ist, sich
also die Einträge jeder Zeile zu 1 summieren.
\end{example}
