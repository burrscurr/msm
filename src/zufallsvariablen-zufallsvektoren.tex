\begin{definition}{Zufallsvariable}{zvar}
Eine Funktion $X: \Omega \to \R$ wird als \defw{Zufallsvariable}
bezeichnet. Die Zufallsvariable ordnet jedem Ereignis einer
\link{def:ealg}{Ereignisalgebra} eine reelle Zahl zu.

Der Wertebereich der Zufallszahl wird als \defw{Zustandsraum} $S = X(\Omega)$
bezeichnet.

Ist $S$ endlich oder abzählbar unendlich, wird die Zufallsvariable als
\defw{diskret}, ist $S$ überabzählbar unendlich als \defw{stetig} bezeichnet.
\end{definition}

\begin{definition}{Wahrscheinlichkeitsverteilung}{verteilung}
Sei $A$ eine \link{def:ealg}{Ereignisalgebra}, $X$ eine \link{def:zvar}{Zufallsvariable}
mit Zustandsraum $S$. Dann heißt Funktion $P: S \rightarrow [0,1]$ definiert
durch
\[
P(A) = P(X^{-1}(A)), A \in S
\]
eine \defw{Wahrscheinlichkeitsverteilung}. Eine Verteilung einer stetigen
Zufallsvariable wird als \defw{stetige Verteilung}, die einer diskreten
Zufallsvariable als \defw{diskrete Verteilung} bezeichnet.
\end{definition}

\begin{definition}{Unabhängigkeit von Zufallsvariablen}{zvar-unabh}
Seien $X: \Omega \to \R$, $Y: \Omega \to \R$ \link{def:zvar}{Zufallsvariablen}. $X,
Y$ heißen \defw{unabhängig}, wenn für $a,b,c,d \in \R$, $a\le b, c\le d$ gilt:
\[
P(a < X\le b,c<Y\le d) = P(a<X\le b)\cdot P(c<Y\le d)
\]
\end{definition}

\section{Zufallsvektoren}

Zufallsvektoren sind Kombinationen von Zufallsvariablen. Grundsätzlich ist die
Anzahl der Zufallsvariablen beliebig; hier werden jedoch primär zweidimensionale
Zufallsvariablen betrachtet.

\begin{definition}{Zufallsvektor}{zvektor}
Seien $X_1, ..., X_n$ \link{def:zvar}{Zufallsvariablen}. Die Zusammenfassung
\[
\zvec{X} = (X_1, ..., X_n)^\T
\]
zu einem Vektor heißt \defw{Zufallsvektor}. Sind alle Komponenten des Vektors
diskret beziehungsweise stetig, heißt der Zufallsvektor diskret beziehungsweise
stetig.
\end{definition}

\subsection{Gemeinsame Verteilung}

Die Werte eines (zweidimensionalen) diskreten Zufallsvektors lassen sich in einer Matrix
zusammenfassen:

\begin{definition}{Gemeinsame Verteilung eines Zufallsvektors}{vert-zvektor}
Sei $\zvec{X} = (X, Y)^\T$ ein diskreter Zufallsvektor, wobei die
Zufallsvariable $X$ die Werte $x_0, x_1, ..., x_n$ und $Y$ die Werte $y_0, y_1, ...,
y_m$ annimmt. Dann bezeichnet die Matrix
\[
(p_{ij})_{i=0,...,n;j=0,...,m} \qquad mit\ p_{ij} = P(X=x_i, Y=y_j)
\]
die \defw{gemeinsame Verteilung} von $\zvec{X}$.

Die Verteilung eines stetigen Zufallsvektors $(X,Y)^\T$ wird durch die gemeinsame
Dichte $\rho_{X,Y}(x,y)$ beschrieben.
\end{definition}

\begin{definition}{Randverteilung}{randv}
Sei $(X,Y)^\T$ ein diskreter Zufallsvektor mit gemeinsamer Verteilung $p$.
Die Summierung von Zeilen bzw. Spalten der Matrix werden als
\defw{Randverteilung} bezeichnet:
\begin{align*}
p_{i.}:=P(X=x_i) = \sum_j p_{ij} \\
p_{.j}:=P(Y=y_j) = \sum_i p_{ij}
\end{align*}
\end{definition}

Analog gilt für stetige Zufallsvektoren:
\begin{definition}{Randdichte}{randd}
Sei $(X,Y)^\T$ ein stetiger Zufallsvektor mit gemeinsamer \link{def:dichte}{Dichte}
$\rho_{(X, Y)}$. Dann werden
\begin{align*}
\rho_X(x) = \int\rho_{(X,Y)}(x,y)\mathrm{d} y\quad x\in\R\\
\rho_Y(y) = \int\rho_{(X,Y)}(x,y)\mathrm{d} x\quad y\in\R
\end{align*}
als \defw{Randdichten} des Zufallsvektors bezeichnet.
\end{definition}

\begin{theorem}{}{randd}
Sei $(X,Y)^\T$ ein Zufallsvektor von unabhängigen Zufallsvariablen $X$ und $Y$
mit zugehörigen Randdichten $\rho_X$ und $\rho_Y$. Dann kann die gemeinsame
Verteilung $\rho_{X,Y}$ berechnet werden durch:
\[
\rho_{X,Y}(x,y) = \rho_X(x)\cdot\rho_Y(y)
\]
\end{theorem}

\subsection{Bedingte Wahrscheinlichkeit}

Analog zur \link{def:bedw}{bedingten Wahrscheinlichkeit} von Ereignissen lässt sich
auch für Zufallsvektoren eine bedingte Wahrscheinlichkeit definieren:

\begin{definition}{Bedingte Wahrscheinlichkeit}{bedwahr}
Sei $\zvec{X} = (X, Y)^\T$ ein diskreter Zufallsvektor mit
\link{def:vert-zvektor}{gemeinsamer Verteilung} $(p_{ij})_{i,j=0,1,...}$. Dann ist
mit
\[
P(Y=y_j|X=x_i) := \frac{P(X=x_i, Y=y_j)}{P(X=x_i)} = \frac{p_{ij}}{p_{i.}}
\]
die \defw{bedingte Wahrscheinlichkeit} von $Y=y_j$ unter Bedingung $X=x_i$ gegeben.
\end{definition}

\begin{definition}{Bedingte Dichte}{bedd}
Ist $\zvec{X}$ ein stetiger Zufallsvektor mit gemeinsamer Dichte
$\rho_{(X,Y)}$, so bezeichnen die Funktionen
\begin{align*}
\rho_{X|Y=y}(x) = \frac{\rho_{X,Y}(x,y)}{\rho_Y(y)}\\
\rho_{Y|X=x}(y) = \frac{\rho_{X,Y}(x,y)}{\rho_X(x)}
\end{align*}
die \defw{bedingte Dichte} von $X$ unter $Y=y$ bzw. $Y$ unter $X=x$.
\end{definition}

Ebenso analog zur \link{def:bedw}{bedingten Wahrscheinlichkeit} von Ereignissen kann
der \link{satz:bayes}{Satz von Bayes} für diskrete bzw. stetige Zufallsvektoren
formuliert werden:
\begin{align*}
p_{ij} = P(Y=y_j|X=x_i)\cdot p_{i.}\\
\rho_{X,Y} = \rho_{Y|X=x}(y)\cdot\rho_X(x)
\end{align*}
